\documentclass[12pt]{article}

%opening
\title{Ground Delay Program Planning: A Dynamic Programming Approach}
\author{Haoxiang Yang}
\date{7/11/2016}
\usepackage[margin=0.5in,tmargin = 1in, bmargin = 1in]{geometry}
\usepackage{enumitem}
\usepackage{mathtools}
\usepackage{mathrsfs}
\usepackage{amsfonts}
\usepackage{algpseudocode}
\usepackage{listings}
\usepackage{amsthm}
\usepackage{grffile}
\usepackage{sidecap}
\usepackage{float}
\usepackage{longtable}
\allowdisplaybreaks

\begin{document}

\maketitle

\section{Introduction}
	In air traffic control, one of the most used and most effective strategy is the ground delay program. Describe what GDP is. List what decisions ATM has to make.\\
	\newline Literature review about: 1. optimizing GDP using static optimization; 2. optimizing GDP using stochastic optimization; 3. optimizing GDP using heuristics; 4. airport capacity estimation/prediction;\\
	\newline Structure of the paper;
\section{Weather Forecasts and Airport Capacity Prediction}\label{Prediction}
	Here we can put stuff about the capacity prediction.
	\begin{itemize}
		\item What factors affect the airport capacity?
		\item The method we used to predict the airport capacity
	\end{itemize}

\section{An Equivalent Stochastic Integer Program Formulation}\label{Model}
	In this section we construct a multi-stage stochastic integer program to model GDP mathematically, using the prediction result from Section~\ref{Prediction} to model the airport capacity. We assume that at each time period, the airport capacity of each following time period is a random variable with the distribution given by the prediction result. \\
	\newline Here we assume the decision make is solving the stochastic program every time period. Suppose we have a finite number of scenarios (even it may be really large) and we denote each scenario as \(\omega \in \Omega\). The uncertainty parameters \(\xi\) include the arrival capacity following the time period \(t\). We can write down the formulation of this multi-stage stochastic integer program of time period \(t\) as follows:
	\begin{longtable}[H]{l l l}
		Set and Parameters & &\\
		\(\mathcal{F} = \mathcal{F}_1 \cup \mathcal{F}_2\) & \(\qquad\)& set of flights\\
		\(\mathcal{F}_1\) & \(\qquad\)& set of flights that are not exempted\\
		\(\mathcal{F}_2\) & \(\qquad\)& set of flights that are exempted\\
		\(\mathcal{F}_3\) & \(\qquad\)& set of flights that originate from the researched airport\\
		\(T\) & \(\qquad\) & length of planning horizon\\
		\(\mathcal{T} = \{1, \dots,T\}\) & \(\qquad\) & set of time periods\\
		\(\Omega\) & \(\qquad\) & set of scenarios\\
		\(B\) & \(\qquad\) & number of scenario partitions \\
		\(\Omega_i\) & \(\qquad\) & set of scenarios in partition \(i = \{1, \dots, B\}\)\\
		\(\tau_f\) & \(\qquad\) & the duration of flight \(f \in \mathcal{F}\)\\
		\(S_f\) & \(\qquad\) & original departure time of flight \(f\in \mathcal{F}\)\\
		\(CA_t^\omega\)  & \(\qquad\) & the arrival capacity of time period \(t\) in scenario \(\omega \in \Omega\)\\
		\(CD_t^\omega\)  & \(\qquad\) & the departure capacity of time period \(t\) in scenario \(\omega \in \Omega\)\\
		\(CT_t^\omega\)  & \(\qquad\) & the total airport capacity of time period \(t\) in scenario \(\omega \in \Omega\)\\
		\(N\) & \(\qquad\) & length of the planning horizon \\
		\(g_{ft}\) & \(\qquad\) & ground delay cost of flight \(f \in \mathcal{F}_1\) at time period \(t\), \\
		\(a_{f}\) & \(\qquad\) & airborne delay cost of flight \(f \in \mathcal{F}\) per period, \\
		\(c_{ft}\) & \(\qquad\) & cancellation cost of flight \(f \in \mathcal{F}_1 \cup \mathcal{F}_3\) at time period \(t\)\\
		\(o_{ft}\) & \(\qquad\) & taxi-out cost of flight \(f \in \mathcal{F}_3\)\\
		\(O_{f}\) & \(\qquad\) & scheduled departure time for flight \(f \in \mathcal{F}_3\)\\
		& &\\
		Decision Variables & &\\
		\(X_{ft}^\omega\) & \(\qquad = \) & \( \begin{cases*}
		1 & if the flight $f$ has been cleared for departure before $t$,\\
		& $\forall f \in \mathcal{F}_1, t \in \{S_f - N, \dots, T\}, \omega \in \Omega$\\
		0 & otherwise\\
		\end{cases*}\)\\
		\(D_{ft}^\omega\) & \(\qquad = \) & \( \begin{cases*}
		1 & if the flight $f$ has departed before $t$, $\forall f \in \mathcal{F}_1, t \in \{S_f, \dots, T\}, \omega \in \Omega$\\
		0 & otherwise\\
		\end{cases*}\)\\
		\(L_{ft}^\omega\) & \(\qquad = \) & \( \begin{cases*}
		1 & if the flight $f$ has arrived before $t$, $\forall f \in \mathcal{F}, t \in \{S_f + \tau_f, \dots, T\}, \omega \in \Omega$\\
		0 & otherwise\\
		\end{cases*}\)\\
		\(Y_{ft}^\omega\) & \(\qquad = \) & \( \begin{cases*}
		1 & if the flight $f$ has landed before $t$, $\forall f \in \mathcal{F}, t \in \{S_f + \tau_f, \dots, T\}, \omega \in \Omega$\\
		0 & otherwise\\
		\end{cases*}\)\\
		\(Z_{ft}^\omega\) & \(\qquad = \) & \( \begin{cases*}
		1 & if the flight $f$ has been cancelled before $t$, $\forall f \in \mathcal{F}_1, t \in \mathcal{T}, \omega \in \Omega$\\
		0 & otherwise\\
		\end{cases*}\)\\
		\(E_{ft}^\omega\) & \(\qquad = \) & \( \begin{cases*}
		1 & if the flight $f$ has taken off before $t$, $\forall f \in \mathcal{F}_3, t \in \mathcal{T}, \omega \in \Omega$\\
		0 & otherwise\\
		\end{cases*}\)\\
		\(EZ_{ft}^\omega\) & \(\qquad = \) & \( \begin{cases*}
		1 & if the flight $f$ has been cancelled before $t$, $\forall f \in \mathcal{F}_3, t \in \mathcal{T}, \omega \in \Omega$\\
		0 & otherwise\\
		\end{cases*}\)	
	\end{longtable}
	\begin{align}
		\min \quad & \sum_{\omega \in \Omega} p^\omega \bigg[ \sum_{f \in \mathcal{F}_1} \left(\sum_{t = S_f}^T g_{ft}(D_{ft}^\omega - D_{f(t-1)}^\omega) + \sum_{t = 1}^T c_{ft}(Z_{ft}^\omega - Z_{f(t-1)}^\omega) \right)\\
		& +\sum_{f \in \mathcal{F}}\sum_{t = S_f+\tau_f}^T a_{f}(L_{ft}^\omega - Y_{ft}^\omega) + \sum_{f \in \mathcal{F}_3} \sum_{t = O_f}^{T} \left(o_{f}(E_{ft}^\omega - E_{f(t-1)}^\omega) + \sum_{f \in \mathcal{F}_3} c_{ft}(EZ_{ft}^\omega - EZ_{f(t-1)}^\omega)\right) \bigg]\\
		\text{s.t.} \quad &  Y_{fT}^\omega + Z_{fT}^\omega = 1 \qquad \qquad \forall f \in \mathcal{F}_1, \omega \in \Omega\\
		& Y_{fT}^\omega = 1 \qquad \qquad \forall f \in \mathcal{F}_2, \omega \in \Omega\\
		& E_{fT}^\omega + EZ_{fT}^\omega = 1 \qquad \qquad \forall f \in \mathcal{F}_3, \omega \in \Omega\\
		& Z_{ft}^\omega + D_{ft}^\omega \leq 1 \qquad \qquad \forall f \in \mathcal{F}_1, t \in \mathcal{T}, \omega \in \Omega\\
		& E_{ft}^\omega + EZ_{ft}^\omega \leq 1 \qquad \qquad \forall f \in \mathcal{F}_3, t \in \mathcal{T}, \omega \in \Omega\\
		& Y_{ft}^\omega \leq L_{ft}^\omega \qquad \qquad \forall f \in \mathcal{F}, t \in \mathcal{T}, \omega \in \Omega\\
		& X_{ft}^\omega = D_{f(t+N)}^\omega \qquad \qquad \forall f \in \mathcal{F}_1, t \in \{1, \dots, T-N\}, \omega \in \Omega\\
		& D_{ft}^\omega = L_{f(t + \tau_f)}^\omega \qquad \qquad f \in \mathcal{F}_1, t \in \{1, \dots, T - \tau_{f}\}, \omega \in \Omega\\
		& \sum_{f \in \mathcal{F}}\left(Y_{ft}^\omega - Y_{f(t-1)}^\omega\right) \leq CA_{t}^\omega \qquad \qquad \forall t \in \mathcal{T}, \omega \in \Omega \\
		& \sum_{f \in \mathcal{F}_3}\left(E_{ft}^\omega - E_{f(t-1)}^\omega\right) \leq CD_{t}^\omega \qquad \qquad \forall t \in \mathcal{T}, \omega \in \Omega \\
		& \sum_{f \in \mathcal{F}}\left(Y_{ft}^\omega - Y_{f(t-1)}^\omega\right) + \sum_{f \in \mathcal{F}_3}\left(E_{ft}^\omega - E_{f(t-1)}^\omega\right) \leq CT_t^\omega \qquad \qquad \forall t \in \mathcal{T}, \omega \in \Omega \\
		& X_{ft}^\omega \geq X_{f(t-1)}^\omega \qquad \qquad \forall f \in \mathcal{F}_1, t \in \{S_f - N, \dots, T\}, \omega \in \Omega\\
		& Y_{ft}^\omega \geq Y_{f(t-1)}^\omega \qquad \qquad \forall f \in \mathcal{F}, t \in \{S_f + \tau_f, \dots, T\}, \omega \in \Omega\\
		& Z_{ft}^\omega \geq Z_{f(t-1)}^\omega \qquad \qquad \forall f \in \mathcal{F}_1, t \in \{1, \dots, T\}, \omega \in \Omega\\
		& E_{ft}^\omega \geq E_{f(t-1)}^\omega \qquad \qquad \forall f \in \mathcal{F}_3, t \in \{1, \dots, T\}, \omega \in \Omega\\
		& EZ_{ft}^\omega \geq EZ_{f(t-1)}^\omega \qquad \qquad \forall f \in \mathcal{F}_3, t \in \{1, \dots, T\}, \omega \in \Omega\\
		& L_{ft}^\omega = 1 \qquad \qquad \forall f \in \mathcal{F}_2, t \in \{S_f+\tau_f, \dots, T\}, \omega \in \Omega\\
		& L_{ft}^\omega = 0 \qquad \qquad \forall f \in \mathcal{F}_2, t \in \{0, \dots, S_f+\tau_f-1\}, \omega \in \Omega\\
		& X_{ft}^\omega = X_{ft}^{i_1} \qquad \qquad \forall f \in \mathcal{F}_1, t \in \{S_f - N, \dots, T\}, i \in \{1, \dots, B\}, \omega \in \Omega_i\\
		& Y_{ft}^\omega = Y_{ft}^{i_1} \qquad \qquad \forall f \in \mathcal{F}, t \in \{S_f + \tau_f, \dots, T\}, i \in \{1, \dots, B\}, \omega \in \Omega_i\\
		& Z_{ft}^\omega = Z_{ft}^{i_1} \qquad \qquad \forall f \in \mathcal{F}, t \in \{1, \dots, T\}, i \in \{1, \dots, B\}, \omega \in \Omega_i
	\end{align}
	%The issue with this model is that the block diagonal property, which is the key to decompose the problem using Benders decomposition, is lost due to the time-lagging constraints on variable \(X\). It does not mean that this model is not solvable, but it is really hard. We have to relax the constraints on \(X\) to obtain the decomposable structure (similar to the philosophy of progressive hedging) and even then we are not sure whether the problem could be solved with an efficient method.
	
\section{Solution Methods}
	The model in Section~\ref{Model} is the extensive formulation of a multistage stochastic integer program. It is a class of non-convex mathematical program that involves uncertainty and it is extremely hard to solve due to its size. To solve this problem, we combine the modeling effort with recently developed optimization tools.\\
	\newline Stochastic dual dynamic program (SDDP) is developed by \cite{pereira1991multi} for a multistage stochastic linear program with exogenous information interstage-independent. The algorithm uses the piecewise linear nature of the cost-to-go function to iteratively generate Benders cuts to approximate it.\\
	\newline \cite{zou2016nested} presents a solution method to the multistage stochastic integer program with only binary variables. Instead of the linear Benders type cut, they propose to solve the Lagrangian relaxation of the stage problem and generate Lagrangian cuts and strengthened Benders cuts. These cuts, accompanied with the integer L-shape cuts proposed by \cite{laporte1993integer}, can significantly improve the computational performance.\\
	\newline Each stage of the problem can be displayed as follows.
	
	
\section{Computational Results}

\nocite{*}
\bibliographystyle{plain}
\bibliography{gdp_model}

\end{document}
