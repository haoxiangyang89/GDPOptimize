\documentclass[12pt]{article}

%opening
\title{A Data-driven Dynamic Approach of Ground Delay Program Optimization}
\author{Haoxiang Yang, Kenneth Kuhn}
\date{7/11/2016}
\usepackage[margin=0.5in,tmargin = 1in, bmargin = 1in]{geometry}
\usepackage{enumitem}
\usepackage{mathtools}
\usepackage{mathrsfs}
\usepackage{amsfonts}
\usepackage{algpseudocode}
\usepackage{listings}
\usepackage{amsthm}
\usepackage{grffile}
\usepackage{sidecap}
\usepackage{float}
\usepackage{longtable}
\usepackage{color}
\usepackage{graphicx}
\usepackage[]{algorithm2e}
\allowdisplaybreaks

\begin{document}

\maketitle

\section{Introduction} \label{Intro}
	Air traffic flow management (ATFM) refers to strategic air traffic control where the goal is to match the demand for air transportation system resources, including airport arrival and departure `slots' and space in sections of airspace, with the available supply looking several hours into the future.  In ATFM, one of the most commonly used strategies is the Ground Delay Program (GDP) (see \cite{ball2007air} for details).  During a GDP in the United States, Federal Aviation Administration (FAA) air traffic managers will ask air carriers to delay flights destined for a capacity constrained destination airport before the flights have taken off from their origin airports.  When announcing the GDP, traffic managers will set the {\bf program rate}, detailing how many arriving flights the constrained airport can accommodate, the {\bf departure scope}, describing which flights will be delayed, and other operational details.  Note that, according to the departure scope, long-distance flights are often {\bf exempted} from a GDP and are not delayed while other flights are delayed.\\
	\newline
	Numerous research articles have been published on the topic of GDP optimization since \cite{odoni} first described the problem in detail in an academic publication some 30 years ago. Researchers have looked into different components of the problem. Authors started by studying static optimization schemes, where the decision of how to allocate ground delay is made at the beginning of the problem's time horizon.  \cite{vranas1994multi} extend the formulation of the static single airport ground holding problem to optimize GDPs at multiple airports. We use the terms ``ground holding problem" and ``ground delay problem" interchangeably; both refer to mathematical programs modeling GDPs. The authors cited above solve three pure 0-1 integer programming formulations and present some insights into the structure of the multi-airport ground holding problem. \cite{ssp} introduce a mathematical program that captures airport and airspace capacity constraints and that has a special structure allowing the program to be solved to optimality when provided with realistic-sized problems.  \cite{lulli2007european} consider the operational differences between air traffic flow management in Europe and the United States. En route capacity is more often binding in European, leading the authors to consider constraints on en route capacity in their version of the ground holding problem. More recently, \cite{bertsimas2011integer} provide a thorough optimization scheme that produces ground holding, rerouting, speed control and airborne holding instructions for each flight affected by a GDP derived from the solution of a large-scale integer program.\\
	\newline
	As research in stochastic optimization has advanced over the past fifty years, transportation scientists have begun to take uncertainty in airport capacity, due to weather and military control etc., into consideration. \cite{terrab} developed an optimal dynamic programming algorithm and various heuristic algorithms for solving deterministic and stochastic versions of the ground holding problem. Dynamic stochastic integer programs have been proposed by multiple groups of researchers to solve this problem, where the decision maker can adjust his, or her, decisions as time passes and with new information about airfield capacity. The airport capacity is considered a random variable that follows some known probability distribution (\cite{richetta1994dynamic}, \cite{mukherjee2005dynamic}, \cite{mukherjee2007dynamic}, \cite{mukherjee2012ground}). Stochastic integer programs are often large-scale, hard-to-solve problems. A few approaches have been explored to reduce the size of stochastic program models of the ground holding problem. \cite{liu2007scenario} and \cite{liu2008scenario} discuss scenario-based optimization, an approach that makes solving the stochastic program tractable. On the other hand, \cite{ball2001collaborative}, \cite{vossen2006optimization}, and \cite{ball2010ground} introduce heuristic principles to generate reasonably good solutions to the ground delay program.\\
	\newline
	One frequently overlooked part of GDP planning involves estimating the capacity of an airport to accept flights. Capacity estimation is important when determining whether a GDP is needed or not, and when setting the program rate of a GDP. Traffic managers will utilize Terminal Area Forecast (TAF) data describing future airport weather conditions when setting program rates.  The capacity estimation problem involves developing a function that translates airport weather conditions into capacity estimates.  Although more sophisticated approaches to capacity estimation may also consider fleet mix at an airport, the capacity estimation problem is, essentially, not a dynamic problem.  Only changes to airport infrastructure are likely to change airport capacity as a function of conditions.  The FAA has published airport capacity estimates for several airports.  Researchers have also developed methods for using empirical data to estimate capacity. For example, \cite{kim} describe a regression model based on meteorological conditions, runway configuration, and fleet mix that produces arrival and departure capacity estimates. \cite{gorripaty} apply Random Survival Forests and add time-of-day as an explanatory variable. \cite{ramnujam} use quantile regression to find Pareto arrival and departure capacity envelopes.\\
	\newline
	Our paper proposes a comprehensive data-driven method to dynamically optimize the single airport ground delay program, including the airport capacity estimation in the optimization process. The structure of the proposed system is displayed in the following figure. Weather and other information that will predict the capacity of the airport will be automatically collected. The prediction algorithm will be run to generate a probability distribution for the airport capacity profile. Using this distribution, the stochastic program will be solved repeatedly to obtain the feasible optimized departure and arrival plans. Air traffic managers will be able to repeatedly: monitor the predicted capacity profile, implement a portion of the schedule provided by the optimization, and update the optimization model based on implemented actions and new information.\\
%	\begin{figure}[H]
%		\centering
%		\includegraphics[width=0.75\textwidth]{SysStruct}
%		\caption{Proposed System Structure}
%		\label{fig:struct}
%	\end{figure}
	\noindent Other than the contribution of a decision making system, our optimization model is based on the concept of the ``planning horizon", which has been ignored by much previous literature but is key to the real-life implementation of a GDP.  Our optimization model also estimates the capacity for departures and arrivals at an airport, allowing for traffic managers to favor one or the other in a manner consistent with real-life operations. To our knowledge, it is also the first paper to discuss computational techniques in solving the large-scale stochastic optimization problem applied to the ground delay problem. \textcolor{red}{I think we have to be super careful here.  Others are definitely thinking that they are discussing computational techniques.  What, precisely, do we do that's unique?}\\
	\newline
	The next section of this article describes how we estimate the arrival capacity of an airport based on weather forecast data.  Since a GDP aims to match demand with capacity, it is essential to describe capacity as accurately and precisely as possible. The subsequent section describes our multi-stage stochastic integer program for the single airport ground delay program. We next propose methods for solving this math program and detail the results of our computational experiments.  We provide a discussion and conclusion, highlighting areas where further work is needed, to complete this article.
	
\section{Weather Forecasts and Airport Capacity Prediction}\label{Capacity}
	\textcolor{blue}{Here we can put stuff about the capacity prediction.
	\begin{itemize}
		\item What factors affect the airport capacity?
		\item State that with this capacity profile built, we could use the predictive analytics tools to generate a probability distribution of capacity.
	\end{itemize}
	}
	As aircraft arrive at and depart from airports runways, they create wake-vortex turbulence in the air.  The FAA has developed a set of temporal and spatial separation standards for aircraft that wish to use a common runway.  The result is that there is a maximum throughput, or a capacity, associated with the airport runway.\\
	\newline
	When weather conditions are good, aircraft can maintain visual contact with the runway they wish to use and with other aircraft using the same runway or other nearby runways.  During periods of inclement weather, it is more difficult for pilots to monitor and precisely control the situation.  Weather also impacts how different runways are used at an airport.  For example, when visibility is good at San Francisco International Airport (SFO), controllers will implement Simultaneous Offset Instrument Approach (SOIA) procedures that allow two aircraft to land at roughly the same time on parallel runways that are only 750 feet apart.  There is often a layer of marine stratus at SFO that limits visibility and leads traffic managers to prohibit SOIA procedures.  Thus airport capacities are a function of weather and particularly visibility and cloud ceiling conditions.

\section{A Dynamic Stochastic Integer Program Formulation}\label{Model}
	In this section we construct a multi-stage stochastic integer program to model a GDP mathematically. The dynamic solving process will be discussed first and a mathematical programming formulation will be displayed. In this paper, the single airport where the optimization is applied is called the {\em target airport}.\\
	% endogenous decsions and problem structure
	\newline As stated in Section~\ref{Intro} Figure~\ref{fig:struct}, the optimization process starts from taking in the flight schedules and capacity distribution. A multi-stage static stochastic optimization model will be run to generate an optimized implementable schedule for the entire remaining time line, considering the stochastic nature of the capacity. The proposed schedule will be reported to airlines and they will make corresponding decisions to schedule flights for taking off and landing for the current time period. Once this schedule is implemented, airlines will report it back to the optimization system and a new round of optimization will be run. The advantage of this dynamic optimization scheme has been discussed in \cite{richetta1994dynamic}. \\
	\newline There are three categories of flights considered in our problem: non-exempted inbound flights that will arrive at the target airport, exempted inbound flights that usually have a long flight length, and outbound flights that will take off from the target airport. 
	\newline
	We divide a typical day into multiple time periods. For each time period, the decisions to make include whether each inbound plane should be cleared to take off in \(N\) time periods, be canceled, or neither and whether each inbound plane that has already taken off should be scheduled to land at this current time period. The decisions to be made for each outbound plane from the affected airport that has not taken off yet include whether the plane should be scheduled to take off or be canceled. We distinguish the concept of arrival at an airport's terminal area from that of landing at an airport in our model to record airborne cost in the airport terminal area. An inbound flight can arrive at the airspace around its destination airport but not be granted permission to land.  Indeed this happens frequently in real life when a GDP is operational.
	\newline
	Here we also set up the concept of \(N\) time periods between the clearance time and the take-off time of a flight. This means in our model, each flight has to be given a clearance time \(N\) time periods before its take-off time. Note that airport surface routing, airport gate management, and air carrier crew assignment are challenging (research and operational) problems in their own right. In this paper we do not attempt to optimize or manage such processes. As long as airport capacity constraints are met, we assume the result is always feasible in terms of surface routing, crew assignment, gate assignment, and other factors on the ground.\\
	\newline % exogenous information
	We assume that at each time period, the airport capacity of each following time period is a random variable with the distribution given by the prediction result. The probability distribution of airport capacity comes from the predictive analytics tool, using the past weather data as input. Suppose we have a finite number of scenarios over the entire timespan (even if it may be really large), since we have discretized the weather condition of each time period into categories, and we denote each scenario as \(\omega \in \Omega\). The uncertainty parameters \(\xi\) include landing capacity, taking-off capacity and total action capacity (landing plus taking-off) of time period \(t\). \textcolor{red}{Does this assume a scenario-based approach?}\\
	\newline % cost information
	Costs of operations include the following items. For a non-exempted inbound flight, it may incur a ground delay cost due to occupancy of resources (crew, gate, etc.), and a cancellation cost for inconvenience caused for travelers and airlines. For every inbound flight it is possible to incur a airborne cost. On the other side of the operation, each outbound flight may be delayed on the ground (taxi out) or canceled, which would incur different types of costs. \textcolor{blue}{How do we decide the costs? We probably need to decide on some concrete numbers based on the past literature.} \textcolor{red}{I'd recommend pulling some numbers from \cite{true_costs}.  There are 1 or 2 other, similar publications with real cost numbers in them.}\\
	\newline We can write down the formulation of this multi-stage stochastic integer program of time period \(t\) as follows:
	\begin{longtable}[H]{l l l}
		Set and Parameters & &\\
		\(\mathcal{F} = \mathcal{F}_1 \cup \mathcal{F}_2\) & \(\qquad\)& set of inbound flights\\
		\(\mathcal{F}_1\) & \(\qquad\)& set of inbound flights that are non-exempted\\
		\(\mathcal{F}_2\) & \(\qquad\)& set of inbound flights that are exempted\\
		\(\mathcal{F}_3\) & \(\qquad\)& set of outbound flights that originate from the researched airport\\
		\(T\) & \(\qquad\) & time horizon\\
		\(\mathcal{T} = \{1, \dots,T\}\) & \(\qquad\) & set of time periods\\
		\(\Omega\) & \(\qquad\) & set of scenarios\\
		\(B\) & \(\qquad\) & number of scenario partitions \\
		\(\Omega_i\) & \(\qquad\) & set of scenarios in partition \(i = \{1, \dots, B\}\)\\
		\(\tau_f\) & \(\qquad\) & the duration of flight \(f \in \mathcal{F}\)\\
		\(S_f\) & \(\qquad\) & original departure time of flight \(f\in \mathcal{F}\)\\
		\(CA_t^\omega\)  & \(\qquad\) & the arrival capacity of time period \(t\) in scenario \(\omega \in \Omega\)\\
		\(CD_t^\omega\)  & \(\qquad\) & the departure capacity of time period \(t\) in scenario \(\omega \in \Omega\)\\
		\(CT_t^\omega\)  & \(\qquad\) & the total airport capacity of time period \(t\) in scenario \(\omega \in \Omega\)\\
		\(N\) & \(\qquad\) & length of the planning horizon \\
		\(g_{ft}\) & \(\qquad\) & ground delay cost of flight \(f \in \mathcal{F}_1\) at time period \(t\), \\
		\(a_{f}\) & \(\qquad\) & airborne delay cost of flight \(f \in \mathcal{F}\) per period, \\
		\(c_{ft}\) & \(\qquad\) & cancellation cost of flight \(f \in \mathcal{F}_1 \cup \mathcal{F}_3\) at time period \(t\)\\
		\(o_{ft}\) & \(\qquad\) & taxi-out cost of flight \(f \in \mathcal{F}_3\)\\
		\(O_{f}\) & \(\qquad\) & scheduled departure time for flight \(f \in \mathcal{F}_3\)\\
		& &\\
		Decision Variables & &\\
		\(X_{ft}^\omega\) & \(\qquad = \) & \( \begin{cases*}
		1 & if the flight $f$ has been cleared for departure before $t$,\\
		& $\forall f \in \mathcal{F}_1, t \in \{S_f - N, \dots, T\}, \omega \in \Omega$\\
		0 & otherwise\\
		\end{cases*}\)\\
		\(D_{ft}^\omega\) & \(\qquad = \) & \( \begin{cases*}
		1 & if the flight $f$ has departed before $t$, $\forall f \in \mathcal{F}_1, t \in \{S_f, \dots, T\}, \omega \in \Omega$\\
		0 & otherwise\\
		\end{cases*}\)\\
		\(L_{ft}^\omega\) & \(\qquad = \) & \( \begin{cases*}
		1 & if the flight $f$ has arrived before $t$, $\forall f \in \mathcal{F}, t \in \{S_f + \tau_f, \dots, T\}, \omega \in \Omega$\\
		0 & otherwise\\
		\end{cases*}\)\\
		\(Y_{ft}^\omega\) & \(\qquad = \) & \( \begin{cases*}
		1 & if the flight $f$ has landed before $t$, $\forall f \in \mathcal{F}, t \in \{S_f + \tau_f, \dots, T\}, \omega \in \Omega$\\
		0 & otherwise\\
		\end{cases*}\)\\
		\(Z_{ft}^\omega\) & \(\qquad = \) & \( \begin{cases*}
		1 & if the flight $f$ has been cancelled before $t$, $\forall f \in \mathcal{F}_1, t \in \mathcal{T}, \omega \in \Omega$\\
		0 & otherwise\\
		\end{cases*}\)\\
		\(E_{ft}^\omega\) & \(\qquad = \) & \( \begin{cases*}
		1 & if the flight $f$ has taken off before $t$, $\forall f \in \mathcal{F}_3, t \in \mathcal{T}, \omega \in \Omega$\\
		0 & otherwise\\
		\end{cases*}\)\\
		\(EZ_{ft}^\omega\) & \(\qquad = \) & \( \begin{cases*}
		1 & if the flight $f$ has been cancelled before $t$, $\forall f \in \mathcal{F}_3, t \in \mathcal{T}, \omega \in \Omega$\\
		0 & otherwise\\
		\end{cases*}\)	
	\end{longtable}
	\begin{align}
		\min \quad & \sum_{\omega \in \Omega} p^\omega \bigg[ \sum_{f \in \mathcal{F}_1} \left(\sum_{t = S_f}^T g_{ft}(D_{ft}^\omega - D_{f(t-1)}^\omega) + \sum_{t = 1}^T c_{ft}(Z_{ft}^\omega - Z_{f(t-1)}^\omega) \right) \nonumber\\
		& +\sum_{f \in \mathcal{F}}\sum_{t = S_f+\tau_f}^T a_{f}(L_{ft}^\omega - Y_{ft}^\omega) + \sum_{f \in \mathcal{F}_3} \sum_{t = O_f}^{T} \left(o_{f}(E_{ft}^\omega - E_{f(t-1)}^\omega) + \sum_{f \in \mathcal{F}_3} c_{ft}(EZ_{ft}^\omega - EZ_{f(t-1)}^\omega)\right) \bigg]\\
		\text{s.t.} \quad &  Y_{fT}^\omega + Z_{fT}^\omega = 1 \qquad \qquad \forall f \in \mathcal{F}_1, \omega \in \Omega \label{cancelORfly}\\
		& Z_{ft}^\omega + D_{ft}^\omega \leq 1 \qquad \qquad \forall f \in \mathcal{F}_1, t \in \mathcal{T}, \omega \in \Omega \label{cancelORplan}\\
		& X_{ft}^\omega = D_{f(t+N)}^\omega \qquad \qquad \forall f \in \mathcal{F}_1, t \in \{1, \dots, T-N\}, \omega \in \Omega \label{planahead}\\
		& D_{ft}^\omega = L_{f(t + \tau_f)}^\omega \qquad \qquad f \in \mathcal{F}_1, t \in \{1, \dots, T - \tau_{f}\}, \omega \in \Omega \label{depLag}\\
		& Y_{fT}^\omega = 1 \qquad \qquad \forall f \in \mathcal{F}_2, \omega \in \Omega \label{mustLand}\\
		& L_{ft}^\omega = 1 \qquad \qquad \forall f \in \mathcal{F}_2, t \in \{S_f+\tau_f, \dots, T\}, \omega \in \Omega \label{mustArr}\\
		& L_{ft}^\omega = 0 \qquad \qquad \forall f \in \mathcal{F}_2, t \in \{0, \dots, S_f+\tau_f-1\}, \omega \in \Omega \label{cannotArr}\\
		& Y_{ft}^\omega \leq L_{ft}^\omega \qquad \qquad \forall f \in \mathcal{F}, t \in \mathcal{T}, \omega \in \Omega \label{arrland}\\
		& E_{fT}^\omega + EZ_{fT}^\omega = 1 \qquad \qquad \forall f \in \mathcal{F}_3, \omega \in \Omega \label{cancelORto}\\
		& E_{ft}^\omega + EZ_{ft}^\omega \leq 1 \qquad \qquad \forall f \in \mathcal{F}_3, t \in \mathcal{T}, \omega \in \Omega \label{cancelTO2}\\
		& \sum_{f \in \mathcal{F}}\left(Y_{ft}^\omega - Y_{f(t-1)}^\omega\right) \leq CA_{t}^\omega \qquad \qquad \forall t \in \mathcal{T}, \omega \in \Omega \label{arrCap}\\
		& \sum_{f \in \mathcal{F}_3}\left(E_{ft}^\omega - E_{f(t-1)}^\omega\right) \leq CD_{t}^\omega \qquad \qquad \forall t \in \mathcal{T}, \omega \in \Omega \label{deptCap}\\
		& \sum_{f \in \mathcal{F}}\left(Y_{ft}^\omega - Y_{f(t-1)}^\omega\right) + \sum_{f \in \mathcal{F}_3}\left(E_{ft}^\omega - E_{f(t-1)}^\omega\right) \leq CT_t^\omega \qquad \qquad \forall t \in \mathcal{T}, \omega \in \Omega \label{totalCap}\\
		& X_{ft}^\omega \geq X_{f(t-1)}^\omega \qquad \qquad \forall f \in \mathcal{F}_1, t \in \{S_f - N, \dots, T\}, \omega \in \Omega \label{xupdate}\\
		& Y_{ft}^\omega \geq Y_{f(t-1)}^\omega \qquad \qquad \forall f \in \mathcal{F}, t \in \{S_f + \tau_f, \dots, T\}, \omega \in \Omega \label{yupdate}\\
		& Z_{ft}^\omega \geq Z_{f(t-1)}^\omega \qquad \qquad \forall f \in \mathcal{F}_1, t \in \{1, \dots, T\}, \omega \in \Omega \label{zupdate}\\
		& E_{ft}^\omega \geq E_{f(t-1)}^\omega \qquad \qquad \forall f \in \mathcal{F}_3, t \in \{1, \dots, T\}, \omega \in \Omega \label{eupdate}\\
		& EZ_{ft}^\omega \geq EZ_{f(t-1)}^\omega \qquad \qquad \forall f \in \mathcal{F}_3, t \in \{1, \dots, T\}, \omega \in \Omega \label{ezupdate}\\
		& X_{ft}^\omega = X_{ft}^{i_1} \qquad \qquad \forall f \in \mathcal{F}_1, t \in \{S_f - N, \dots, T\}, i \in \{1, \dots, B\}, \omega \in \Omega_i \label{nonAntX}\\
		& Y_{ft}^\omega = Y_{ft}^{i_1} \qquad \qquad \forall f \in \mathcal{F}, t \in \{S_f + \tau_f, \dots, T\}, i \in \{1, \dots, B\}, \omega \in \Omega_i \label{nonAntY}\\
		& Z_{ft}^\omega = Z_{ft}^{i_1} \qquad \qquad \forall f \in \mathcal{F}, t \in \{1, \dots, T\}, i \in \{1, \dots, B\}, \omega \in \Omega_i \label{nonAntZ}
	\end{align}
	The objective of the optimization problem is to minimize expected total cost, including costs for each inbound and outbound flight. For a non-exempted inbound flight, it would either land or be canceled at the end of the time span (Constraint (\ref{cancelORfly})), and it cannot take off when it is canceled (Constraint (\ref{cancelORplan})). Constraint (\ref{planahead}) and (\ref{depLag}) determine the separation between a non-exempted flight's clearance time, departure time and arrival time. Constraint (\ref{mustLand}) - (\ref{cannotArr}) specify arrival/landing time limitations for exempted inbound flights. An inbound flight will have to arrive at the airport before landing due to Constraint (\ref{arrland}).\\
	\newline
	For an outbound flight, it should either take off or be cancelled, which is stated in Constraint (\ref{cancelORto}) and (\ref{cancelTO2}). Constraint (\ref{arrCap}) - (\ref{totalCap}) specify the capacity limitation, which is characterized by the convex envelope described in Section~\ref{Capacity}.\\
	\newline
	The rest of the constraints are structural constraints. Constraint (\ref{xupdate}) - (\ref{ezupdate}) state the transition of state variables \(X, Y, Z, E\) and \(EZ\) from time period \(t\) to \(t+1\). Constraint (\ref{nonAntX}) - (\ref{nonAntZ}) are called ``non-anticipitivatiy" constraints. These are a type of important constraints that state the decision can only be made based on the known information.  
	
\section{Solution Methods}
	The model in Section~\ref{Model} is an extensive formulation of a multistage stochastic integer program. It is a class of non-convex mathematical program that involves uncertainty and it is extremely hard to solve due to its size. We can see all constraints are scenario specific, while the number of scenarios is growing exponentially in \(T\). Even with a modest estimation with two scenarios per time period and 48 time period a day, the number of scenarios is \(2^{48} \approx 2.81\times 10^{14}\). One way to tackle this problem is reduce the number of scenarios. Recently \cite{liu2008scenario} propose a method to classify scenarios into a small number of nominal groups and offer a braching identification procedure in order to perform optimization dynamically.\\
	\newline
	We take another approach to solve this problem by combining modeling effort with state-of-art decomposition tools to deal with problems with modest number of scenario per time period and a long time horizon. Stochastic dual dynamic program (SDDP) is developed by \cite{pereira1991multi} for a multistage stochastic linear program with exogenous information interstage-independent. The algorithm uses the piecewise linear nature of the cost-to-go function to iteratively generate Benders cuts to approximate it.\\
	\newline \cite{zou2016nested} presents a solution method to the multistage stochastic integer program with only binary variables. Instead of the linear Benders type cut, they propose to solve the Lagrangian relaxation of the stage problem and generate Lagrangian cuts and strengthened Benders cuts. These cuts, accompanied with the integer L-shape cuts proposed by \cite{laporte1993integer}, can significantly improve the computational performance.\\
	\newline At time period \(t\), let us denote the variables that record the state of being cleared for each flight be \(\mathbf{H}_{ft}\), which is a vector of 0/1 to record whether flight \(f\) has taken off. This vector has a length of \(N + \tau_f\) in order to keep a full history. In addition, we should establish state variables for landing \(V_{ft}\), departure \(W_{ft}\) and cancellation \(U_{ft}\). We can write down the problem for the time period \(t\) as follows:
	\begin{align}
		(M_t)\quad \min \quad & \sum_{f \in \mathcal{F}_1} \left[ g_{ft} (D_f - HD_f) + c_{ft} (Z_f - HZ_f) \right] + \sum_{f \in \mathcal{F}} \left[ a_f (L_f - Y_f)\right] + & \nonumber \\
		& \sum_{f \in \mathcal{F}_3} \left[ o_f(E_f - HE_f) + c_{ft}(EZ_f - HEZ_f)\right] + \sum_{\omega \in \Omega_t} p^\omega \theta^\omega &\\
		\text{s.t.} \quad & Z_f + D_f \leq 1 & \forall f \in \mathcal{F}_1\\
		& E_f + EZ_f \leq 1 & \forall f \in \mathcal{F}_3\\
		& Y_f \leq HL_f & \forall f \in \mathcal{F}\\
		& \sum_{f \in \mathcal{F}} Y_f - HY_f \leq CA_t &\\
		& \sum_{f \in \mathcal{F}_3} E_f + HE_f \leq CD_t &\\
		& \sum_{f \in \mathcal{F}} (Y_f - HY_f) + \sum_{f \in \mathcal{F}_3} (E_f + HE_f) \leq CT_t &\\
		& X_f \geq HX_f & \forall f \in \mathcal{F}_1\\
		& Z_f \geq HZ_f & \forall f \in \mathcal{F}_1\\
		& Y_f \geq HY_f & \forall f \in \mathcal{F}\\
		& E_f \geq HE_f & \forall f \in \mathcal{F}_3\\
		& EZ_f \geq HEZ_f & \forall f \in \mathcal{F}_3\\
		& \text{Make a local copy of the states variables}& \label{inheritCon}\\
		& \text{Record the state variables}& \label{passDown}\\
		& \theta^\omega \geq \sum_{f \in \mathcal{F}_1} \mathbf{\pi}_{H,f}^{\omega,k,t,f} (\mathbf{H}_{ft} - \hat{\mathbf{H}}_{ft}) + \sum_{f \in \mathcal{F}}\pi_{V}^{\omega,k,t,f} (V_{ft} - \hat{V}_{ft}) & \nonumber\\
		& + \sum_{f \in \mathcal{F}_3} \pi_{W}^{\omega,k,t,f} (W_{ft} - \hat{W}_{ft}) + \sum_{f \in \mathcal{F}_1 \cup \mathcal{F}_3} \pi_{U}^{\omega,k,t,f} (U_{ft} - \hat{U}_{ft}) + \hat{Q}^{\omega,k,t} & \forall \omega \in \Omega_t
	\end{align}
	In this problem, \(HX,HD,HY,HL,HZ,HE,HEZ\) are all local variables. Constraint Set (\ref{inheritCon}) is used to set up the local copy in the following way:
	\begin{align*}
		& HX_f = H_{f,(t-1),1}& \forall f \in \mathcal{F}_1\\
		& HD_f = H_{f,(t-1),N}& \forall f \in \mathcal{F}_1\\
		& HY_f = V_{f,(t-1)}& \forall f \in \mathcal{F}\\
		& HL_f = H_{f,(t-1)}& \forall f \in \mathcal{F}\\
		& HZ_f = U_{f,(t-1)}& \forall f \in \mathcal{F}_1\\
		& HE_f = W_{f,(t-1)}& \forall f \in \mathcal{F}_3\\
		& HEZ_f = U_{f,(t-1)}& \forall f \in \mathcal{F}_3\\
	\end{align*}
	To proceed to the next time period \(t+1\), the state variable should be updated using the decisions in the time period \(t\), which is specified in Constraint Set (\ref{passDown}):
	\begin{align*}
		& H_{f,t,1} = X_f & \forall f \in \mathcal{F}_1\\
		& H_{f,t,i} = H_{f,t-1,(i-1)} & \forall f \in \mathcal{F}_1, i = 2, \dots, N+\tau_f\\
		& V_{f,t} = Y_f & \forall f \in \mathcal{F}\\
		& W_{f,t} = E_f & \forall f \in \mathcal{F}_3\\
		& U_{f,t} = Z_f & \forall f \in \mathcal{F}_1\\
		& U_{f,t} = EZ_f & \forall f \in \mathcal{F}_3
	\end{align*}
	The algorithm to solve this multi-stage stochastic integer program is displayed as follows:\\
	\begin{algorithm}[H]
		
	\end{algorithm}
	\noindent If we implement a vanilla version of SDDiP to solve this problem, we will solve a problem with 48 stages, with each stage being a large-scale MIP. 
\section{Experimental Results}

\nocite{*}
\bibliographystyle{plain}
\bibliography{gdp_model}

\end{document}
