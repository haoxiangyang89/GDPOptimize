\documentclass[12pt]{article}

%opening
\title{Ground Delay Program Planning: A Dynamic Programming Approach}
\author{Haoxiang Yang, Kenneth Kuhn}
\date{7/11/2016}
\usepackage[margin=0.5in,tmargin = 1in, bmargin = 1in]{geometry}
\usepackage{enumitem}
\usepackage{mathtools}
\usepackage{mathrsfs}
\usepackage{amsfonts}
\usepackage{algpseudocode}
\usepackage{listings}
\usepackage{amsthm}
\usepackage{grffile}
\usepackage{sidecap}
\usepackage{float}
\usepackage{longtable}
\usepackage{color}
\usepackage[]{algorithm2e}
\allowdisplaybreaks

\begin{document}

\maketitle

\section{Introduction}
	Air traffic flow management (ATFM) refers to strategic or high-level air traffic control where the goal is to match the demand for air transportation system resources, including airport arrival and departure `slots' and space in sections of airspace, with the available supply.  In ATFM, one of the most common and most effective strategies is the Ground Delay Program (GDP).  During a GDP in the United States, Federal Aviation Administration (FAA) air traffic managers will ask air carriers to delay flights destined for a capacity constrained destination airport before the flights have taken off from their origin airports.  When announcing the GDP, traffic managers will set the {\em program rate}, detailing how many arriving flights the constrained airport can accommodate, the {\em departure scope}, describing which flights will be delayed, and other operational details.  Note that, according to the departure scope, long-distance flights are often {\em exempted} from a GDP and are not delayed while other flights are delayed.\\
	\newline
	\textcolor{blue}{Literature review about: 1. optimizing GDP using static optimization; 2. optimizing GDP using stochastic optimization; 3. optimizing GDP using heuristics; 4. airport capacity estimation/prediction;}\\
	\newline
	The next section of this article describes how we estimate the arrival capacity of an airport based on weather forecast data.  Since a GDP aims to match demand with capacity, it is essential to describe capacity as accurately and precisely as possible.  The subsequent section describes our multi-stage stochastic integer program modeling the impacts of a Ground Delay Program.  We next propose methods for solving this math program and detail the results of our computational experiments.  We provide a discussion and conclusion, highlighting areas where further work is needed, to complete this article.
	
\section{Weather Forecasts and Airport Capacity Prediction}\label{Prediction}
	\textcolor{blue}{Here we can put stuff about the capacity prediction.
	\begin{itemize}
		\item What factors affect the airport capacity?
		\item The method we used to predict the airport capacity
	\end{itemize}
	}
	As aircraft arrive at and depart from airports runways, they create wake-vortex turbulence in the air.  The FAA has developed a set of temporal and spatial separation standards for aircraft that wish to use a common runway.  The result is that there is a maximum throughput, or a capacity, associated with the airport runway.\\
	\newline
	When weather conditions are benign, aircraft can maintain visual contact with the runway they wish to use and other aircraft using the same runway or other runways.  During periods of inclement weather, it is more difficult for pilots to monitor and precisely control the situation.  Weather also impacts how different runways are used at an airport.  For example, when visibility is good at San Francisco International Airport (SFO), controllers will implement Simultaneous Offset Instrument Approach (SOIA) procedures that allow two aircraft to land at roughly the same time on parallel runways that are only 750 feet apart.  There is often a layer of marine stratus at SFO that limits visibility and leads traffic managers to prohibit SOIA procedures.\\
	\newline\textcolor{red}{Jeremy Eckhause at RAND is actually providing me with a 2-3 page write up about capacity estimation using empirical data.  He promised to have it to me by the end of the year.  We could plug that in here if you like.  But $\dots$ he told me that his conclusion might be that the best course of action is to use the FAA Capacity Profiles.  So that's not ideal.  Maybe we can use part of this write-up, assuming you're ok with adding him as a co-author and he's ok if we don't use the FAA Capacity Profiles when we define capacity.}

\section{An Equivalent Stochastic Integer Program Formulation}\label{Model}
	In this section we construct a multi-stage stochastic integer program to model a GDP mathematically. The structure of the problem will be discussed first and a mathematical programming formulation will be displayed. In this paper, the single airport where the optimization is applied is called the {\em target airport}.\\
	% endogenous decsions and problem structure
	\newline There are three categories of flights considered in our problem: non-exempted inbound flights that will arrive at the target airport, exempted inbound flights that usually have a long flight length, and outbound flights that will take off from the target airport. \textcolor{blue}{Last time period.} We divide a typical day into 48 time periods (using a half an hour as a time period). For each time period, the decisions to make at each time period include whether an inbound plane should be cleared to take off in \(N\) time periods, or be canceled, whether an arriving plane should land at this current time period, and whether an outbound plane should take off or be canceled. We distinguish  arrival from landing in our model to record airborne cost. An inbound flight can arrive at the airport but could not be granted permission to land. Here we set up the concept of \(N\) time periods between the clearance time and the take-off time for a flight to approximate the reality. This means in our model, each flight has to be given a clearance time \(N\) time periods before its take-off time. There are plenty of work done for the airport runway/gate management. In this paper we do not look into this micro-scope management, which means as long as the capacity constraints are met, we assume it is always feasible in terms of crews, gates or other factors on the ground.\\
	\newline % exogenous information
	We assume that at each time period, the airport capacity of each following time period is a random variable with the distribution given by the prediction result. The probability distribution comes from Section~\ref{Prediction}. Suppose we have a finite number of scenarios over the entire timespan (even it may be really large), since we have discretized the weather condition of each time period into categories, and we denote each scenario as \(\omega \in \Omega\). The uncertainty parameters \(\xi\) include landing capacity, taking-off capacity and total action capacity (landing plus taking-off) of time period \(t\). \\
	\newline % cost information
	Costs of operations include the following items. For a non-exempted inbound flight, it may incur a ground delay cost due to occupancy of resources (crew, gate, etc.), and a cancellation cost for inconvenience caused for travelers and airlines. For every inbound flight it is possible to incur a airborne cost. On the other side of the operation, each outbound flight may be delayed on the ground (taxi out) or canceled, which would incur different types of costs. \textcolor{blue}{How do we decide the costs? We probably need to decide on some concrete numbers based on the past literature.} \\
	\newline We can write down the formulation of this multi-stage stochastic integer program of time period \(t\) as follows:
	\begin{longtable}[H]{l l l}
		Set and Parameters & &\\
		\(\mathcal{F} = \mathcal{F}_1 \cup \mathcal{F}_2\) & \(\qquad\)& set of inbound flights\\
		\(\mathcal{F}_1\) & \(\qquad\)& set of inbound flights that are non-exempted\\
		\(\mathcal{F}_2\) & \(\qquad\)& set of inbound flights that are exempted\\
		\(\mathcal{F}_3\) & \(\qquad\)& set of outbound flights that originate from the researched airport\\
		\(T\) & \(\qquad\) & time horizon\\
		\(\mathcal{T} = \{1, \dots,T\}\) & \(\qquad\) & set of time periods\\
		\(\Omega\) & \(\qquad\) & set of scenarios\\
		\(B\) & \(\qquad\) & number of scenario partitions \\
		\(\Omega_i\) & \(\qquad\) & set of scenarios in partition \(i = \{1, \dots, B\}\)\\
		\(\tau_f\) & \(\qquad\) & the duration of flight \(f \in \mathcal{F}\)\\
		\(S_f\) & \(\qquad\) & original departure time of flight \(f\in \mathcal{F}\)\\
		\(CA_t^\omega\)  & \(\qquad\) & the arrival capacity of time period \(t\) in scenario \(\omega \in \Omega\)\\
		\(CD_t^\omega\)  & \(\qquad\) & the departure capacity of time period \(t\) in scenario \(\omega \in \Omega\)\\
		\(CT_t^\omega\)  & \(\qquad\) & the total airport capacity of time period \(t\) in scenario \(\omega \in \Omega\)\\
		\(N\) & \(\qquad\) & length of the planning horizon \\
		\(g_{ft}\) & \(\qquad\) & ground delay cost of flight \(f \in \mathcal{F}_1\) at time period \(t\), \\
		\(a_{f}\) & \(\qquad\) & airborne delay cost of flight \(f \in \mathcal{F}\) per period, \\
		\(c_{ft}\) & \(\qquad\) & cancellation cost of flight \(f \in \mathcal{F}_1 \cup \mathcal{F}_3\) at time period \(t\)\\
		\(o_{ft}\) & \(\qquad\) & taxi-out cost of flight \(f \in \mathcal{F}_3\)\\
		\(O_{f}\) & \(\qquad\) & scheduled departure time for flight \(f \in \mathcal{F}_3\)\\
		& &\\
		Decision Variables & &\\
		\(X_{ft}^\omega\) & \(\qquad = \) & \( \begin{cases*}
		1 & if the flight $f$ has been cleared for departure before $t$,\\
		& $\forall f \in \mathcal{F}_1, t \in \{S_f - N, \dots, T\}, \omega \in \Omega$\\
		0 & otherwise\\
		\end{cases*}\)\\
		\(D_{ft}^\omega\) & \(\qquad = \) & \( \begin{cases*}
		1 & if the flight $f$ has departed before $t$, $\forall f \in \mathcal{F}_1, t \in \{S_f, \dots, T\}, \omega \in \Omega$\\
		0 & otherwise\\
		\end{cases*}\)\\
		\(L_{ft}^\omega\) & \(\qquad = \) & \( \begin{cases*}
		1 & if the flight $f$ has arrived before $t$, $\forall f \in \mathcal{F}, t \in \{S_f + \tau_f, \dots, T\}, \omega \in \Omega$\\
		0 & otherwise\\
		\end{cases*}\)\\
		\(Y_{ft}^\omega\) & \(\qquad = \) & \( \begin{cases*}
		1 & if the flight $f$ has landed before $t$, $\forall f \in \mathcal{F}, t \in \{S_f + \tau_f, \dots, T\}, \omega \in \Omega$\\
		0 & otherwise\\
		\end{cases*}\)\\
		\(Z_{ft}^\omega\) & \(\qquad = \) & \( \begin{cases*}
		1 & if the flight $f$ has been cancelled before $t$, $\forall f \in \mathcal{F}_1, t \in \mathcal{T}, \omega \in \Omega$\\
		0 & otherwise\\
		\end{cases*}\)\\
		\(E_{ft}^\omega\) & \(\qquad = \) & \( \begin{cases*}
		1 & if the flight $f$ has taken off before $t$, $\forall f \in \mathcal{F}_3, t \in \mathcal{T}, \omega \in \Omega$\\
		0 & otherwise\\
		\end{cases*}\)\\
		\(EZ_{ft}^\omega\) & \(\qquad = \) & \( \begin{cases*}
		1 & if the flight $f$ has been cancelled before $t$, $\forall f \in \mathcal{F}_3, t \in \mathcal{T}, \omega \in \Omega$\\
		0 & otherwise\\
		\end{cases*}\)	
	\end{longtable}
	\begin{align}
		\min \quad & \sum_{\omega \in \Omega} p^\omega \bigg[ \sum_{f \in \mathcal{F}_1} \left(\sum_{t = S_f}^T g_{ft}(D_{ft}^\omega - D_{f(t-1)}^\omega) + \sum_{t = 1}^T c_{ft}(Z_{ft}^\omega - Z_{f(t-1)}^\omega) \right) \nonumber\\
		& +\sum_{f \in \mathcal{F}}\sum_{t = S_f+\tau_f}^T a_{f}(L_{ft}^\omega - Y_{ft}^\omega) + \sum_{f \in \mathcal{F}_3} \sum_{t = O_f}^{T} \left(o_{f}(E_{ft}^\omega - E_{f(t-1)}^\omega) + \sum_{f \in \mathcal{F}_3} c_{ft}(EZ_{ft}^\omega - EZ_{f(t-1)}^\omega)\right) \bigg]\\
		\text{s.t.} \quad &  Y_{fT}^\omega + Z_{fT}^\omega = 1 \qquad \qquad \forall f \in \mathcal{F}_1, \omega \in \Omega \label{cancelORfly}\\
		& Z_{ft}^\omega + D_{ft}^\omega \leq 1 \qquad \qquad \forall f \in \mathcal{F}_1, t \in \mathcal{T}, \omega \in \Omega \label{cancelORplan}\\
		& X_{ft}^\omega = D_{f(t+N)}^\omega \qquad \qquad \forall f \in \mathcal{F}_1, t \in \{1, \dots, T-N\}, \omega \in \Omega \label{planahead}\\
		& D_{ft}^\omega = L_{f(t + \tau_f)}^\omega \qquad \qquad f \in \mathcal{F}_1, t \in \{1, \dots, T - \tau_{f}\}, \omega \in \Omega \label{depLag}\\
		& Y_{fT}^\omega = 1 \qquad \qquad \forall f \in \mathcal{F}_2, \omega \in \Omega \label{mustLand}\\
		& L_{ft}^\omega = 1 \qquad \qquad \forall f \in \mathcal{F}_2, t \in \{S_f+\tau_f, \dots, T\}, \omega \in \Omega \label{mustArr}\\
		& L_{ft}^\omega = 0 \qquad \qquad \forall f \in \mathcal{F}_2, t \in \{0, \dots, S_f+\tau_f-1\}, \omega \in \Omega \label{cannotArr}\\
		& Y_{ft}^\omega \leq L_{ft}^\omega \qquad \qquad \forall f \in \mathcal{F}, t \in \mathcal{T}, \omega \in \Omega \label{arrland}\\
		& E_{fT}^\omega + EZ_{fT}^\omega = 1 \qquad \qquad \forall f \in \mathcal{F}_3, \omega \in \Omega\\
		& E_{ft}^\omega + EZ_{ft}^\omega \leq 1 \qquad \qquad \forall f \in \mathcal{F}_3, t \in \mathcal{T}, \omega \in \Omega\\
		& \sum_{f \in \mathcal{F}}\left(Y_{ft}^\omega - Y_{f(t-1)}^\omega\right) \leq CA_{t}^\omega \qquad \qquad \forall t \in \mathcal{T}, \omega \in \Omega \\
		& \sum_{f \in \mathcal{F}_3}\left(E_{ft}^\omega - E_{f(t-1)}^\omega\right) \leq CD_{t}^\omega \qquad \qquad \forall t \in \mathcal{T}, \omega \in \Omega \\
		& \sum_{f \in \mathcal{F}}\left(Y_{ft}^\omega - Y_{f(t-1)}^\omega\right) + \sum_{f \in \mathcal{F}_3}\left(E_{ft}^\omega - E_{f(t-1)}^\omega\right) \leq CT_t^\omega \qquad \qquad \forall t \in \mathcal{T}, \omega \in \Omega \\
		& X_{ft}^\omega \geq X_{f(t-1)}^\omega \qquad \qquad \forall f \in \mathcal{F}_1, t \in \{S_f - N, \dots, T\}, \omega \in \Omega\\
		& Y_{ft}^\omega \geq Y_{f(t-1)}^\omega \qquad \qquad \forall f \in \mathcal{F}, t \in \{S_f + \tau_f, \dots, T\}, \omega \in \Omega\\
		& Z_{ft}^\omega \geq Z_{f(t-1)}^\omega \qquad \qquad \forall f \in \mathcal{F}_1, t \in \{1, \dots, T\}, \omega \in \Omega\\
		& E_{ft}^\omega \geq E_{f(t-1)}^\omega \qquad \qquad \forall f \in \mathcal{F}_3, t \in \{1, \dots, T\}, \omega \in \Omega\\
		& EZ_{ft}^\omega \geq EZ_{f(t-1)}^\omega \qquad \qquad \forall f \in \mathcal{F}_3, t \in \{1, \dots, T\}, \omega \in \Omega\\
		& X_{ft}^\omega = X_{ft}^{i_1} \qquad \qquad \forall f \in \mathcal{F}_1, t \in \{S_f - N, \dots, T\}, i \in \{1, \dots, B\}, \omega \in \Omega_i\\
		& Y_{ft}^\omega = Y_{ft}^{i_1} \qquad \qquad \forall f \in \mathcal{F}, t \in \{S_f + \tau_f, \dots, T\}, i \in \{1, \dots, B\}, \omega \in \Omega_i\\
		& Z_{ft}^\omega = Z_{ft}^{i_1} \qquad \qquad \forall f \in \mathcal{F}, t \in \{1, \dots, T\}, i \in \{1, \dots, B\}, \omega \in \Omega_i
	\end{align}
	The objective of the optimization problem is to minimize expected total cost, including costs for each inbound and outbound flight. For a non-exempted inbound flight, it would either land or be canceled at the end of the time span (Constraint (\ref{cancelORfly})), and it cannot take off when it is canceled (Constraint (\ref{cancelORplan})). Constraint (\ref{planahead}) and (\ref{depLag}) determine its clearance time, departure time and arrival time. Constraint \ref{mustLand} - \ref{cannotArr} specify arrival/landing time limitations for exempted inbound flights.
	
\section{Solution Methods}
	The model in Section~\ref{Model} is the extensive formulation of a multistage stochastic integer program. It is a class of non-convex mathematical program that involves uncertainty and it is extremely hard to solve due to its size. To solve this problem, we combine the modeling effort with recently developed optimization tools.\\
	\newline Stochastic dual dynamic program (SDDP) is developed by \cite{pereira1991multi} for a multistage stochastic linear program with exogenous information interstage-independent. The algorithm uses the piecewise linear nature of the cost-to-go function to iteratively generate Benders cuts to approximate it.\\
	\newline \cite{zou2016nested} presents a solution method to the multistage stochastic integer program with only binary variables. Instead of the linear Benders type cut, they propose to solve the Lagrangian relaxation of the stage problem and generate Lagrangian cuts and strengthened Benders cuts. These cuts, accompanied with the integer L-shape cuts proposed by \cite{laporte1993integer}, can significantly improve the computational performance.\\
	\newline At time period \(t\), let us denote the variables that record the state of being cleared for each flight be \(\mathbf{H}_{ft}\), which is a vector of 0/1 to record whether flight \(f\) has taken off. This vector has a length of \(N + \tau_f\) in order to keep a full history. In addition, we should establish state variables for landing \(V_{ft}\), departure \(W_{ft}\) and cancellation \(U_{ft}\). We can write down the problem for the time period \(t\) as follows:
	\begin{align}
		(M_t)\quad \min \quad & \sum_{f \in \mathcal{F}_1} \left[ g_{ft} (D_f - HD_f) + c_{ft} (Z_f - HZ_f) \right] + \sum_{f \in \mathcal{F}} \left[ a_f (L_f - Y_f)\right] + & \nonumber \\
		& \sum_{f \in \mathcal{F}_3} \left[ o_f(E_f - HE_f) + c_{ft}(EZ_f - HEZ_f)\right] + \sum_{\omega \in \Omega_t} p^\omega \theta^\omega &\\
		\text{s.t.} \quad & Z_f + D_f \leq 1 & \forall f \in \mathcal{F}_1\\
		& E_f + EZ_f \leq 1 & \forall f \in \mathcal{F}_3\\
		& Y_f \leq HL_f & \forall f \in \mathcal{F}\\
		& \sum_{f \in \mathcal{F}} Y_f - HY_f \leq CA_t &\\
		& \sum_{f \in \mathcal{F}_3} E_f + HE_f \leq CD_t &\\
		& \sum_{f \in \mathcal{F}} (Y_f - HY_f) + \sum_{f \in \mathcal{F}_3} (E_f + HE_f) \leq CT_t &\\
		& X_f \geq HX_f & \forall f \in \mathcal{F}_1\\
		& Z_f \geq HZ_f & \forall f \in \mathcal{F}_1\\
		& Y_f \geq HY_f & \forall f \in \mathcal{F}\\
		& E_f \geq HE_f & \forall f \in \mathcal{F}_3\\
		& EZ_f \geq HEZ_f & \forall f \in \mathcal{F}_3\\
		& \text{Make a local copy of the states variables}& \label{inheritCon}\\
		& \text{Record the state variables}& \label{passDown}\\
		& \theta^\omega \geq \sum_{f \in \mathcal{F}_1} \mathbf{\pi}_{H,f}^{\omega,k,t,f} (\mathbf{H}_{ft} - \hat{\mathbf{H}}_{ft}) + \sum_{f \in \mathcal{F}}\pi_{V}^{\omega,k,t,f} (V_{ft} - \hat{V}_{ft}) & \nonumber\\
		& + \sum_{f \in \mathcal{F}_3} \pi_{W}^{\omega,k,t,f} (W_{ft} - \hat{W}_{ft}) + \sum_{f \in \mathcal{F}_1 \cup \mathcal{F}_3} \pi_{U}^{\omega,k,t,f} (U_{ft} - \hat{U}_{ft}) + \hat{Q}^{\omega,k,t} & \forall \omega \in \Omega_t
	\end{align}
	In this problem, \(HX,HD,HY,HL,HZ,HE,HEZ\) are all local variables. Constraint Set (\ref{inheritCon}) is used to set up the local copy in the following way:
	\begin{align*}
		& HX_f = H_{f,(t-1),1}& \forall f \in \mathcal{F}_1\\
		& HD_f = H_{f,(t-1),N}& \forall f \in \mathcal{F}_1\\
		& HY_f = V_{f,(t-1)}& \forall f \in \mathcal{F}\\
		& HL_f = H_{f,(t-1)}& \forall f \in \mathcal{F}\\
		& HZ_f = U_{f,(t-1)}& \forall f \in \mathcal{F}_1\\
		& HE_f = W_{f,(t-1)}& \forall f \in \mathcal{F}_3\\
		& HEZ_f = U_{f,(t-1)}& \forall f \in \mathcal{F}_3\\
	\end{align*}
	To proceed to the next time period \(t+1\), the state variable should be updated using the decisions in the time period \(t\), which is specified in Constraint Set (\ref{passDown}):
	\begin{align*}
		& H_{f,t,1} = X_f & \forall f \in \mathcal{F}_1\\
		& H_{f,t,i} = H_{f,t-1,(i-1)} & \forall f \in \mathcal{F}_1, i = 2, \dots, N+\tau_f\\
		& V_{f,t} = Y_f & \forall f \in \mathcal{F}\\
		& W_{f,t} = E_f & \forall f \in \mathcal{F}_3\\
		& U_{f,t} = Z_f & \forall f \in \mathcal{F}_1\\
		& U_{f,t} = EZ_f & \forall f \in \mathcal{F}_3
	\end{align*}
	The algorithm to solve this multi-stage stochastic integer program is displayed as follows:\\
	\begin{algorithm}[H]
		
	\end{algorithm}
	\noindent If we implement a vanilla version of SDDiP to solve this problem, we will solve a problem with 48 stages, with each stage being a large-scale MIP. 
\section{Computational Results}

\nocite{*}
\bibliographystyle{plain}
\bibliography{gdp_model}

\end{document}
