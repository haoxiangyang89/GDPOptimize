\documentclass[12pt]{article}

%opening
\title{A Dynamic, Data-driven Approach for Ground Delay Program Optimization}
\author{Haoxiang Yang, Kenneth Kuhn}
\date{8/7/2016}
\usepackage[margin=0.5in,tmargin = 1in, bmargin = 1in]{geometry}
\usepackage{enumitem}
\usepackage{mathtools}
\usepackage{mathrsfs}
\usepackage{amsfonts}
\usepackage{algpseudocode}
\usepackage{listings}
\usepackage{amsthm}
\usepackage{grffile}
\usepackage{sidecap}
\usepackage{float}
\usepackage{longtable}
\usepackage{color}
\usepackage{graphicx}
\usepackage{hyperref}
\usepackage{algorithm}
\allowdisplaybreaks

\begin{document}

\maketitle

\section{Introduction} \label{Intro}
	Air traffic flow management (ATFM) refers to strategic air traffic control where the goal is to match the demand for air transportation system resources, including airport arrival and departure `slots' and space in sections of airspace, with the available supply looking several hours into the future.  In ATFM, one of the most commonly used strategies is the Ground Delay Program (GDP) (see \cite{ball2007air} for details).  During a GDP in the United States, Federal Aviation Administration (FAA) air traffic managers will ask air carriers to delay flights destined for a capacity constrained destination airport before the flights have taken off from their origin airports.  When announcing the GDP, traffic managers will set the {\bf program rate}, detailing how many arriving flights the constrained airport can accommodate, the {\bf departure scope}, describing which flights will be delayed, and other operational details.  Note that, according to the departure scope, long-distance flights are often {\bf exempted} from a GDP and are not delayed while other flights are delayed.\\
	\newline
	Numerous research articles have been published on the topic of GDP optimization since \cite{odoni} first described the problem in detail in an academic publication some 30 years ago. Researchers have looked into different components of the problem. Authors started by studying static optimization schemes, where the decision of how to allocate ground delay is made at the beginning of the problem's time horizon.  \cite{vranas1994multi} extend the formulation of the static single airport ground holding problem to optimize GDPs at multiple airports. We use the terms ``ground holding problem" and ``ground delay problem" interchangeably; both refer to mathematical programs modeling GDPs. The authors cited above solve three pure 0-1 integer programming formulations and present some insights into the structure of the multi-airport ground holding problem. \cite{ssp} introduce a mathematical program that captures airport and airspace capacity constraints and that has a special structure allowing the program to be solved to optimality when provided with realistic-sized problems.  \cite{lulli2007european} consider the operational differences between air traffic flow management in Europe and the United States. En route capacity is more often binding in European, leading the authors to consider constraints on en route capacity in their version of the ground holding problem. More recently, \cite{bertsimas2011integer} provide a thorough optimization scheme that produces ground holding, rerouting, speed control and airborne holding instructions for each flight affected by a GDP derived from the solution of a large-scale integer program.\\
	\newline
	As research in stochastic optimization has advanced over the past fifty years, transportation scientists have begun to take uncertainty in airport capacity, due to weather and military control etc., into consideration. \cite{terrab} developed an optimal dynamic programming algorithm and various heuristic algorithms for solving deterministic and stochastic versions of the ground holding problem. Dynamic stochastic integer programs have been proposed by multiple groups of researchers to solve this problem, where the decision maker can adjust his or her decisions as time passes and with new information about airfield capacity. The airport capacity is considered a random variable that follows some known probability distribution (\cite{richetta1994dynamic}, \cite{mukherjee2005dynamic}, \cite{mukherjee2007dynamic}, \cite{mukherjee2012ground}). Stochastic integer programs are often large-scale, hard-to-solve problems. A few approaches have been explored to reduce the size of stochastic program models of the ground holding problem. \cite{liu2007scenario} and \cite{liu2008scenario} discuss scenario-based optimization, an approach that makes solving the stochastic program tractable. On the other hand, \cite{ball2001collaborative}, \cite{vossen2006optimization}, and \cite{ball2010ground} introduce heuristic principles to generate reasonably good solutions to the ground delay program.\\
	\newline
	One frequently overlooked part of GDP planning involves estimating the capacity of an airport to accept flights. Capacity estimation is important when determining whether a GDP is needed or not, and when setting the program rate of a GDP. Traffic managers will utilize Terminal Area Forecast (TAF) data describing future airport weather conditions when setting program rates.  The capacity estimation problem involves developing a function that translates airport weather conditions into capacity estimates.  Although more sophisticated approaches to capacity estimation may also consider fleet mix at an airport, the capacity estimation problem is, essentially, not a dynamic problem.  Only changes to airport infrastructure are likely to change airport capacity as a function of conditions.  The FAA has published airport capacity estimates for several airports.  Researchers have also developed methods for using empirical data to estimate capacity. Examples include \cite{kim}, \cite{gorripaty}, and \cite{ramnujam}, which are all discussed in a later section of this article.\\
	\newline
	Our paper proposes a comprehensive data-driven method to dynamically optimize a single airport ground delay program, including airport capacity estimation in the optimization process. The structure of the proposed system is displayed in the following figure. Weather and other information that will predict the capacity of the airport will be automatically collected. The prediction algorithm will be run to generate a probability distribution for the airport capacity profile. Using this distribution, the stochastic program will be solved repeatedly to obtain the feasible optimized departure and arrival plans. Air traffic managers will be able to repeatedly: monitor the predicted capacity profile, implement a portion of the schedule provided by the optimization, and update the optimization model based on implemented actions and new information.\\
	%\begin{figure}[H]
	%	\centering
	%	\includegraphics[width=0.75\textwidth]{SysStruct}
	%	\caption{Proposed System Structure}
	%	\label{fig:struct}
	%\end{figure}
	\newline
	\noindent Other than the contribution of a decision making system, our optimization model is based on the concept of the ``planning horizon," which is key to the real-life implementation of a GDP.  Our optimization model also estimates the capacity for departures and arrivals at an airport, allowing for traffic managers to favor one or the other in a manner consistent with real-life operations. Further, our work involves solving a large-scale multi-stage stochastic optimization problem for the ground delay problem using novel decomposition techniques. \\
	\newline
	The next section of this article describes how we estimate the arrival capacity of an airport based on weather forecast data.  Since a GDP aims to match demand with capacity, it is essential to describe capacity as accurately and precisely as possible. The subsequent section describes our multi-stage stochastic integer program for the single airport ground delay program. We next propose methods for solving this math program and detail the results of our computational experiments.  We provide a discussion and conclusion, highlighting areas where further work is needed, to complete this article.
	
\section{Weather Forecasts and Airport Capacity Prediction}\label{Capacity}
	As aircraft arrive at and depart from airport runways, they create wake-vortex turbulence in the air.  The FAA has developed a set of temporal and spatial separation standards for aircraft that wish to use a common runway in order to ensure safe operations.  When weather conditions are good, pilots can maintain visual contact with the runway they wish to use and with other aircraft using the same runway or other nearby runways.  This allows aircraft to land at times separated by intervals not much larger than wake vortex separation standards.  During periods of inclement weather, it is more difficult for pilots to monitor and precisely control the situation.  Weather also impacts how different runways are used at an airport.  For example, when visibility is good at San Francisco International Airport (SFO), controllers will implement Simultaneous Offset Instrument Approach (SOIA) procedures that allow two aircraft to land at roughly the same time on parallel runways that are only 750 feet apart.  There is often a layer of marine stratus at SFO that limits visibility and leads traffic managers to prohibit SOIA procedures.  More generally, runway configurations in use at airports, which describe which runways are used for arrival and departure traffic, change with wind and weather conditions.  There is often a trade-off between arrival and departure traffic.  In summary there is a time-varying maximum arrival and departure throughput curve (a capacity envelope) associated with an airport.   Airport capacity envelopes are a function of weather and particularly visibility and cloud ceiling conditions.  It is, however, not obvious what the capacity envelope of an airport is or will be particularly given weather forecast uncertainty.\\
	\newline
	The FAA has published Airport Capacity Profile documents online\footnote{\url{https://www.faa.gov/airports/planning_capacity/profiles/}} for 35 of the busiest airports in the country. The documents were developed by the Mitre Corporation as part of work undertaken for the 2014 update of the Airport Capacity Benchmark Report.  The documents include arrival-departure capacity envelopes for three or four sets of meteorological conditions including cases where ``Visual,'' ``Marginal,'' and ``Instrument'' flight rules would be in effect. These conditions are (separately) defined in terms of ceiling and visibility in each airport's Capacity Profile. Some profiles also include separate capacity envelopes for when arriving or departing traffic is prioritized.  For example, one curve passing through the (labeled) point (121,95) described the current capacity of Hartsfield-Jackson Atlanta International Airport (ATL), in terms of hourly arrivals and hourly departures under ``Visual Weather Conditions'' when there is an ``Arrival Priority.''\\
	\newline
	The FAA profiles appear to be the best publicly available source of information on airport capacities.  As was mentioned in the Introduction, a few researchers have developed methodologies for estimating airport capacities using observed arrival and departure counts and various covariate data.  \cite{gorripaty} introduce an approach based on Random Survival Forests and covariate data including several weather variables and the hour of the day for estimating arrival capacity per hour.  \cite{kim} develop a regression model of arrival and departure rates per quarter hour based on meteorological condition, runway configuration, called rates, and fleet mix.  Unfortunately, much of the data the authors use is not publicly available or available at all when looking several hours into the future making it difficult to apply the authors' methods in a GDP planning context.  \cite{ramnujam} provide details on a quantile regression based approach that the authors use to estimate arrival-departure capacity envelopes per quarter hour.\\
	\newline
	\textcolor{red}{Our method for estimating capacity is based on. . . }
	\textcolor{blue}{
	\begin{itemize}
			\item Describe the data scraped from the Internet.
			\item Why we build 40 categories?
			\item Describe the categories and how they are built.
			\item An example of the categories.
			\item Test the validity of categories.
	\end{itemize}}

\section{A Dynamic Stochastic Integer Program Formulation}\label{Model}
	In this section we construct a multi-stage stochastic integer program to model a GDP mathematically. We discuss the overall framework of our proposed method for planning a GDP first before providing the relevant mathematical programming formulation.  In this paper, the single airport where the optimization is applied is called the {\em target airport}.\\
	% endogenous decsions and problem structure
	\newline As stated in Section~\ref{Intro} Figure~\ref{fig:struct}, our optimization process begins with flight schedules and stochastic models of target airport capacities. A multi-stage static stochastic optimization model will be run to generate an optimized implementable schedule, considering the stochastic nature of the capacity. The proposed schedule will be reported to airlines and they will make corresponding decisions to schedule flights for taking off and landing for the current time period. Once this schedule is implemented, traffic managers and our system will track developments and begin a new round of optimization. The advantage of a dynamic optimization scheme like this for GDP planning has been discussed in \cite{richetta1994dynamic}. \\
	\newline There are three categories of flights considered in our problem: non-exempted inbound flights that will arrive at the target airport, exempted inbound flights that usually have a long flight length, and outbound flights that will take off from the target airport. \\
	\newline
	We divide a typical day into multiple time periods. For each time period, the decisions to make include whether each inbound plane should be cleared to take off in \(N\) time periods, be canceled, or neither and whether each inbound plane that has already taken off should be scheduled to land at this current time period. The decisions to be made for each outbound plane from the affected airport that has not taken off yet include whether the plane should be scheduled to take off or be canceled. We distinguish the concept of arrival at an airport's terminal area from that of landing at an airport in our model to record airborne cost in the airport terminal area. An inbound flight can arrive at the airspace around its destination airport but not be granted permission to land.  Indeed this happens frequently in real life when a GDP is operational.
	\newline
	Here we also set up the concept of \(N\) time periods between the clearance time and the take-off time of a flight. This means in our model, each flight has to be given a clearance time \(N\) time periods before its take-off time. Note that airport surface routing, airport gate management, and air carrier crew assignment are challenging (research and operational) problems in their own right. In this paper we do not attempt to optimize or manage such processes. As long as airport capacity constraints are met, we assume the result is always feasible in terms of surface routing, crew assignment, gate assignment, and other factors on the ground.\\
	\newline % exogenous information
	We assume that at each time period, the airport capacity of each following time period is a random variable with the distribution given by the prediction result. The probability distribution of airport capacity comes from the predictive analytics tool, using forecast weather data as input. Suppose we have a finite number of scenarios over the entire timespan (even if it may be really large), since we have discretized the weather condition of each time period into categories, and we denote each scenario as \(\omega \in \Omega\). The uncertainty parameters \(\xi\) include landing capacity, taking-off capacity and total action capacity (landing plus taking-off) of time period \(t\).\\
	\newline % cost information
	Our model attempts to minimize air carrier costs.  We model these costs using the results of more detailed studies focused on this topic.  For example, \cite{true_costs} quantified the total costs, to air carriers, of an additional minute of ground or airborne delay as a function of numerous relevant variables such as aircraft type.  The authors note that in addition to fuel costs, airlines face maintenance / engineering costs, flight crew costs, cabin crew costs, airport charges, and passenger delay and handling charges all of which can be broken into charges that will and will not change due to individual flight delays.  Indirect operating costs such as ground equipment charges and passenger service staff salaries can also be impacted by flight delays (Ibid.).  Two of the authors of the cited study later used the results in the context of ATFM, exploring the trade-off between ground and airborne delay \cite{cook}.  The studies mentioned above were based on data from European air carriers.  One recent effort developed a similar airline cost model for American air carriers \cite{ferguson}.  Results provided by the authors reveal the costs of delay in July 2007 broken out by air carrier and per-minute of delay, and as well as by time-of-day and phase-of-flight.\\
	\newline We can write down the formulation of this multi-stage stochastic integer program of time period \(t\) using the following notation for sets and parameters of interest.
	\begin{longtable}[H]{l l l}
		Set and Parameters & &\\
		\(\mathcal{F} = \mathcal{F}_1 \cup \mathcal{F}_2\) & \(\qquad\)& set of inbound flights\\
		\(\mathcal{F}_1\) & \(\qquad\)& set of inbound flights that are non-exempted\\
		\(\mathcal{F}_2\) & \(\qquad\)& set of inbound flights that are exempted\\
		\(\mathcal{F}_3\) & \(\qquad\)& set of outbound flights that originate from the target airport\\
		\(T\) & \(\qquad\) & time horizon\\
		\(\mathcal{T} = \{1, \dots,T\}\) & \(\qquad\) & set of time periods\\
		\(\Omega\) & \(\qquad\) & set of scenarios\\
		\(B\) & \(\qquad\) & number of scenario partitions \\
		\(\Omega_i\) & \(\qquad\) & set of scenarios in partition \(i = \{1, \dots, B\}\)\\
		\(\tau_f\) & \(\qquad\) & the duration of flight \(f \in \mathcal{F}\)\\
		\(S_f\) & \(\qquad\) & original departure time of flight \(f\in \mathcal{F}\)\\
		\(CA_t^\omega\)  & \(\qquad\) & the arrival capacity of time period \(t\) in scenario \(\omega \in \Omega\)\\
		\(CD_t^\omega\)  & \(\qquad\) & the departure capacity of time period \(t\) in scenario \(\omega \in \Omega\)\\
		\(CT_t^\omega\)  & \(\qquad\) & the total airport capacity of time period \(t\) in scenario \(\omega \in \Omega\)\\
		\(N\) & \(\qquad\) & length of the planning horizon \\
		\(g_{ft}\) & \(\qquad\) & ground delay cost of flight \(f \in \mathcal{F}_1\) at time period \(t\), \\
		\(a_{f}\) & \(\qquad\) & airborne delay cost of flight \(f \in \mathcal{F}\) per period, \\
		\(c_{ft}\) & \(\qquad\) & cancellation cost of flight \(f \in \mathcal{F}_1 \cup \mathcal{F}_3\) at time period \(t\)\\
		\(o_{ft}\) & \(\qquad\) & taxi-out cost of flight \(f \in \mathcal{F}_3\)\\
		\(O_{f}\) & \(\qquad\) & scheduled departure time for flight \(f \in \mathcal{F}_3\)\\
	\end{longtable}
	\noindent
	Our mathematical program employs seven (five?) sets of decision variables capturing whether or not each flight has done various things {\bf before time $t$}, including: being cleared for departure, departing, arriving at the target airport's terminal area, landing, and being cancelled.\\
	\newline\textcolor{red}{Can we combine the EZ and Z decision variables?  And maybe the E and D decision variables?  I see that they relate to different sets of flights but I think they represent the same thing and it might help us simplify the notation.}\\
	\newline The decision variables are as follows.\\
	\begin{longtable}[H]{l l l}
		Decision Variables & &\\
		\(X_{ft}^\omega\) & \(\qquad = \) & \( \begin{cases*}
		1 & if the flight $f$ has been cleared for departure before $t$,\\
		& $\forall f \in \mathcal{F}_1, t \in \{S_f - N, \dots, T\}, \omega \in \Omega$\\
		0 & otherwise\\
		\end{cases*}\)\\
		\(D_{ft}^\omega\) & \(\qquad = \) & \( \begin{cases*}
		1 & if the flight $f$ has departed before $t$, $\forall f \in \mathcal{F}_1, t \in \{S_f, \dots, T\}, \omega \in \Omega$\\
		0 & otherwise\\
		\end{cases*}\)\\
		\(L_{ft}^\omega\) & \(\qquad = \) & \( \begin{cases*}
		1 & if the flight $f$ has arrived before $t$, $\forall f \in \mathcal{F}, t \in \{S_f + \tau_f, \dots, T\}, \omega \in \Omega$\\
		0 & otherwise\\
		\end{cases*}\)\\
		\(Y_{ft}^\omega\) & \(\qquad = \) & \( \begin{cases*}
		1 & if the flight $f$ has landed before $t$, $\forall f \in \mathcal{F}, t \in \{S_f + \tau_f, \dots, T\}, \omega \in \Omega$\\
		0 & otherwise\\
		\end{cases*}\)\\
		\(Z_{ft}^\omega\) & \(\qquad = \) & \( \begin{cases*}
		1 & if the flight $f$ has been cancelled before $t$, $\forall f \in \mathcal{F}_1, t \in \mathcal{T}, \omega \in \Omega$\\
		0 & otherwise\\
		\end{cases*}\)\\
		\(E_{ft}^\omega\) & \(\qquad = \) & \( \begin{cases*}
		1 & if the flight $f$ has taken off before $t$, $\forall f \in \mathcal{F}_3, t \in \mathcal{T}, \omega \in \Omega$\\
		0 & otherwise\\
		\end{cases*}\)\\
		\(EZ_{ft}^\omega\) & \(\qquad = \) & \( \begin{cases*}
		1 & if the flight $f$ has been cancelled before $t$, $\forall f \in \mathcal{F}_3, t \in \mathcal{T}, \omega \in \Omega$\\
		0 & otherwise\\
		\end{cases*}\)	
	\end{longtable}
%	\noindent
%	Figure~\ref{fig:exampleDV} shows example graphs of reasonable values of decision variables along with description of what each set of graphs would represent.\\
	%\begin{figure}[H]
	%	\centering
	%	\includegraphics[width=\textwidth]{exampleDV}
	%	\caption{Examples of Decision Variables}
	%	\label{fig:exampleDV}
	%\end{figure}
	\noindent Our extensive formulation of the GDP optimization problem is shown as follows.
	\begin{subequations}
		\begin{align}
			\min \quad & \sum_{\omega \in \Omega} p^\omega \bigg[ \sum_{f \in \mathcal{F}_1} \left(\sum_{t = S_f}^T g_{ft}(D_{ft}^\omega - D_{f(t-1)}^\omega) + \sum_{t = 1}^T c_{ft}(Z_{ft}^\omega - Z_{f(t-1)}^\omega) \right) \nonumber\\
			& +\sum_{f \in \mathcal{F}}\sum_{t = S_f+\tau_f}^T a_{f}(L_{ft}^\omega - Y_{ft}^\omega) + \sum_{f \in \mathcal{F}_3} \sum_{t = O_f}^{T} \left(o_{f}(E_{ft}^\omega - E_{f(t-1)}^\omega) + c_{ft}(EZ_{ft}^\omega - EZ_{f(t-1)}^\omega)\right) \bigg]\\
			\text{s.t.} \quad &  Y_{fT}^\omega + Z_{fT}^\omega = 1 \qquad \qquad \forall f \in \mathcal{F}_1, \omega \in \Omega \label{cancelORfly}\\
			& Z_{ft}^\omega + D_{ft}^\omega \leq 1 \qquad \qquad \forall f \in \mathcal{F}_1, t \in \mathcal{T}, \omega \in \Omega \label{cancelORplan}\\
			& X_{ft}^\omega = D_{f(t+N)}^\omega \qquad \qquad \forall f \in \mathcal{F}_1, t \in \{1, \dots, T-N\}, \omega \in \Omega \label{planahead}\\
			& D_{ft}^\omega = L_{f(t + \tau_f)}^\omega \qquad \qquad f \in \mathcal{F}_1, t \in \{1, \dots, T - \tau_{f}\}, \omega \in \Omega \label{depLag}\\
			& Y_{fT}^\omega = 1 \qquad \qquad \forall f \in \mathcal{F}_2, \omega \in \Omega \label{mustLand}\\
			& L_{ft}^\omega = 1 \qquad \qquad \forall f \in \mathcal{F}_2, t \in \{S_f+\tau_f, \dots, T\}, \omega \in \Omega \label{mustArr}\\
			& L_{ft}^\omega = 0 \qquad \qquad \forall f \in \mathcal{F}_2, t \in \{0, \dots, S_f+\tau_f-1\}, \omega \in \Omega \label{cannotArr}\\
			& Y_{ft}^\omega \leq L_{ft}^\omega \qquad \qquad \forall f \in \mathcal{F}, t \in \mathcal{T}, \omega \in \Omega \label{arrland}\\
			& E_{fT}^\omega + EZ_{fT}^\omega = 1 \qquad \qquad \forall f \in \mathcal{F}_3, \omega \in \Omega \label{cancelORto}\\
			& E_{ft}^\omega + EZ_{ft}^\omega \leq 1 \qquad \qquad \forall f \in \mathcal{F}_3, t \in \mathcal{T}, \omega \in \Omega \label{cancelTO2}\\
			& \sum_{f \in \mathcal{F}}\left(Y_{ft}^\omega - Y_{f(t-1)}^\omega\right) \leq CA_{t}^\omega \qquad \qquad \forall t \in \mathcal{T}, \omega \in \Omega \label{arrCap}\\
			& \sum_{f \in \mathcal{F}_3}\left(E_{ft}^\omega - E_{f(t-1)}^\omega\right) \leq CD_{t}^\omega \qquad \qquad \forall t \in \mathcal{T}, \omega \in \Omega \label{deptCap}\\
			& \sum_{f \in \mathcal{F}}\left(Y_{ft}^\omega - Y_{f(t-1)}^\omega\right) + \sum_{f \in \mathcal{F}_3}\left(E_{ft}^\omega - E_{f(t-1)}^\omega\right) \leq CT_t^\omega \qquad \qquad \forall t \in \mathcal{T}, \omega \in \Omega \label{totalCap}\\
			& X_{ft}^\omega \geq X_{f(t-1)}^\omega \qquad \qquad \forall f \in \mathcal{F}_1, t \in \{S_f - N, \dots, T\}, \omega \in \Omega \label{xupdate}\\
			& Y_{ft}^\omega \geq Y_{f(t-1)}^\omega \qquad \qquad \forall f \in \mathcal{F}, t \in \{S_f + \tau_f, \dots, T\}, \omega \in \Omega \label{yupdate}\\
			& Z_{ft}^\omega \geq Z_{f(t-1)}^\omega \qquad \qquad \forall f \in \mathcal{F}_1, t \in \{1, \dots, T\}, \omega \in \Omega \label{zupdate}\\
			& E_{ft}^\omega \geq E_{f(t-1)}^\omega \qquad \qquad \forall f \in \mathcal{F}_3, t \in \{1, \dots, T\}, \omega \in \Omega \label{eupdate}\\
			& EZ_{ft}^\omega \geq EZ_{f(t-1)}^\omega \qquad \qquad \forall f \in \mathcal{F}_3, t \in \{1, \dots, T\}, \omega \in \Omega \label{ezupdate}\\
			& X_{ft}^\omega = X_{ft}^{i_1} \qquad \qquad \forall f \in \mathcal{F}_1, t \in \{S_f - N, \dots, T\}, i \in \{1, \dots, B\}, \omega \in \Omega_i \label{nonAntX}\\
			& Y_{ft}^\omega = Y_{ft}^{i_1} \qquad \qquad \forall f \in \mathcal{F}, t \in \{S_f + \tau_f, \dots, T\}, i \in \{1, \dots, B\}, \omega \in \Omega_i \label{nonAntY}\\
			& Z_{ft}^\omega = Z_{ft}^{i_1} \qquad \qquad \forall f \in \mathcal{F}, t \in \{1, \dots, T\}, i \in \{1, \dots, B\}, \omega \in \Omega_i \label{nonAntZ}
		\end{align}
	\end{subequations}
	The objective of the optimization problem is to minimize expected total cost, including costs for each inbound and outbound flight. For a non-exempted inbound flight, it would either land or be canceled at the end of the time span (Constraint (\ref{cancelORfly})), and it cannot take off when it is canceled (Constraint (\ref{cancelORplan})). Constraint (\ref{planahead}) and (\ref{depLag}) determine the separation between a non-exempted flight's clearance time, departure time and arrival time. Constraint (\ref{mustLand}) - (\ref{cannotArr}) specify arrival/landing time limitations for exempted inbound flights. An inbound flight will have to arrive at the airport before landing due to Constraint (\ref{arrland}).\\
	\newline
	For an outbound flight, it should either take off or be cancelled, which is stated in Constraint (\ref{cancelORto}) and (\ref{cancelTO2}). Constraint (\ref{arrCap}) - (\ref{totalCap}) specify the capacity limitation, which is characterized by the convex envelope described in Section~\ref{Capacity}.\\
	\newline
	The rest of the constraints are structural constraints. Constraint (\ref{xupdate}) - (\ref{ezupdate}) state the transition of state variables \(X, Y, Z, E\) and \(EZ\) from time period \(t\) to \(t+1\). Constraint (\ref{nonAntX}) - (\ref{nonAntZ}) are called ``non-anticipativity" constraints. These are a type of important constraints that state the decision can only be made based on the known information.  
	
\section{Solution Methods}
	The model in Section~\ref{Model} is an extensive formulation of a multistage stochastic integer program. This is a class of non-convex mathematical programs that involve uncertainty and are typically difficult to solve, particularly for relatively large problem instances. In the formulation introduced in the previous section, all constraints are scenario specific, while the number of scenarios grows exponentially with \(T\). Even for a modest problem statement with two scenarios per time period and 48 time period a day, the number of scenarios is \(2^{48} \approx 2.81\times 10^{14}\). One way to tackle this problem is reduce the number of scenarios. Recently \cite{liu2008scenario} propose a method to classify scenarios into a small number of nominal groups and offer a braching identification procedure in order to perform optimization dynamically.\\
	\newline
	We take another approach to solve this problem, combining our modeling effort with state-of-art decomposition tools which have been designed to deal with problems with a modest number of scenarios per time period but a long time horizon. Stochastic dual dynamic program (SDDP) was developed by \cite{pereira1991multi} for a multistage stochastic linear program where the process by which data are generated exhibits stagewise independence \textcolor{red}{(this needs a bit more clarification)}. The algorithm uses the piecewise linear nature of the cost-to-go function to iteratively generate Benders cuts to approximate it.\\
	\newline \cite{zou2016nested} presents a solution method to the multistage stochastic integer program with only binary variables. Instead of the linear Benders type cut, they propose to solve the Lagrangian relaxation of the stage problem and generate Lagrangian cuts and strengthened Benders cuts. These cuts, accompanied with the integer L-shape cuts proposed by \cite{laporte1993integer}, can significantly improve the computational performance of solution algorithms.\\
	%\newline 
	%At time period \(t\), let us denote the variables that record the state of being cleared for each flight be \(\mathbf{H}_{ft}\), which is a vector of 0/1 to record whether flight \(f\) has taken off. This vector has a length of \(N + \tau_f\) in order to keep a full history. In addition, we should establish state variables for landing \(V_{ft}\), departure \(W_{ft}\) and cancellation \(U_{ft}\). We can write down the problem for the time period \(t\) as follows:\\
	\newline
	\textcolor{red}{(I think the approach needs more of an introduction.  It might help make the formulation more understandable.  Right now I don't get what the $\theta^\omega$ terms are or how the formulation is handling uncertainty.  I'm not sure why constraint sets (35) and (36) are part of the formulation and not pre- or post-processing steps, etc.  I will try to read up on SDDP again - but I think we have to make the text more clear.)}\\
	\newline
	\textcolor{blue}{I agree with the comments. I will change the texts this week to match the model.} We can model the GDP problem as a multi-stage stochastic integer program as follows: 
	\begin{equation}
		z = \min_{x_1 \in X_1} \  f_1(x_1) + \mathbb{E} \left[ \min_{x_2 \in X_2(x_1)} \ f_2(x_1, \xi_2) + \dots + \mathbb{E} \left[ \min_{x_T \in X_T(x_{T-1})} \ f_T(x_{T-1}, \xi_T)\right] \right],
	\end{equation}
	where each function \(f_t, t = 1, \dots T\) is a linear function of the decision variables \(x_t\) at stage \(t\). For each stage the decision variable \(x_t = (D,L,Y,Z,E,EZ,HD,HZ,HY,HX)\), which contains the current stage decisions, status of flights and history records. The uncertainty vector \(\xi_t\) contains the capacity information \(\xi_t = (CA^{\omega_t}, CD^{\omega_t}, CT^{\omega_t})\). We can represent this model with the following transformation:
	\begin{equation}
		z = \min_{x_1 \in X_1} f_1(x_1) + \mathbb{E}\left[ Q_2(x_1, \xi_2)\right],
	\end{equation}
	and for \(t = 2, 3, \dots T - 1\), we have 
	\begin{equation}
		Q_t(x_{t-1},\xi_t) = \min_{x_t \in X_{t}(x_{t-1})}f_t(x_t) + \mathbb{E} \left[ Q_{t+1}(x_t, \xi_{t+1})\right],
	\end{equation}
	Then we can expand the formulation of \(Q_t,\ \forall t = 2, \dots, T-1\) in the following manner.
	\begin{subequations}
		\begin{align}
		Q_t(x_{t-1}, \xi_t)  = & &\\
		\min \quad & \sum_{f \in \mathcal{F}_1} \left[ g_{ft} (D_{ft} - HD_{ft}) + c_{ft} (Z_{ft} - HZ_{ft}) \right]  + & \nonumber \\
		& \sum_{f \in \mathcal{F}_3} \left[ o_f(E_{ft} - HE_{ft}) + c_{ft}(EZ_{ft} - HEZ_{ft}) \right]& \nonumber\\ 
		& +\sum_{f \in \mathcal{F}} \left[ a_f (L_{ft} - Y_{ft})\right]  + \mathbb{E}\left[Q_{t+1}(x_{t},\xi_{t+1}) \right] &\\
		\text{s.t.} \quad & Z_{ft} + D_{ft} \leq 1 & \forall f \in \mathcal{F}\\
		& E_{ft} + EZ_{ft} \leq 1 & \forall f \in \mathcal{F}_3\\
		& Y_{ft} \leq L_{ft} & \forall f \in \mathcal{F}\\
		& D_{ft} = HX_{f(t-1)N} & \forall f \in \mathcal{F}\\
		& L_{ft} = HX_{f(t-1)(N+\tau_{f})}& f \in \mathcal{F}\\
		& HD_{ft} = D_{f(t-1)} & f \in \mathcal{F}\\
		& HE_{ft} = E_{f(t-1)} & f \in \mathcal{F}_3\\
		& HEZ_{ft} = EZ_{f(t-1)} & f \in \mathcal{F}_3\\
		& HX_{fti} = HX_{f(t-1)(i-1)} & f \in \mathcal{F}, i = 2, \dots, \tau_{f} + N\\
		& HY_{ft} = Y_{f(t-1)} & f \in \mathcal{F}\\
		& HZ_{ft} = Z_{f(t-1)} & f \in \mathcal{F}\\
		& \sum_{f \in \mathcal{F}} Y_{ft} - HY_{ft} \leq CA^{\omega_t} &\\
		& \sum_{f \in \mathcal{F}_3} E_{ft} + HE_{ft} \leq CD^{\omega_t} &\\
		& \sum_{f \in \mathcal{F}} (Y_{ft} - HY_{ft}) + \sum_{f \in \mathcal{F}_3} (E_{ft} + HE_{ft}) \leq CT^{\omega_t} &\\
		& X_{ft} \geq HX_{ft} & \forall f \in \mathcal{F}\\
		& Z_{ft} \geq HZ_{ft} & \forall f \in \mathcal{F}\\
		& Y_{ft} \geq HY_{ft} & \forall f \in \mathcal{F}\\
		& E_{ft} \geq HE_{ft} & \forall f \in \mathcal{F}_3\\
		& EZ_{ft} \geq HEZ_{ft} & \forall f \in \mathcal{F}_3\\
		& HX_{fti} \geq HX_{ft(i+1)} & \forall f \in \mathcal{F}, i = 1, \dots, \tau_{f} + N -1,
		\end{align}
	\end{subequations}
	while \(Q_T\) is slightly different to model the terminal conditions. Here to guarantee a relatively complete recourse, we introduce a penalty cost for flights that are not landed or canceled. This cost should be very large to discourage plans that do not finish the action of all flights within the planning horizon.
	\begin{subequations}
		\begin{align}
		\min \quad & \sum_{f \in \mathcal{F}_1} \left[ g_{fT} (D_{fT} - HD_{fT}) + c_{fT} (Z_{fT} - HZ_{fT}) \right] + & \nonumber \\
		& \sum_{f \in \mathcal{F}_3} \left[ o_f(E_{fT} - HE_{fT}) + c_{fT}(EZ_{fT} - HEZ_{fT}) + M RE_{f} \right]& \nonumber\\ 
		& +\sum_{f \in \mathcal{F}} \left[ a_f (L_{fT} - Y_{fT}) + M R_{f}\right]  &\\
		\text{s.t.} \quad & Z_{fT} + D_{fT} \leq 1 & \forall f \in \mathcal{F}\\
		& E_{fT} + EZ_{fT} \leq 1 & \forall f \in \mathcal{F}_3\\
		& Y_{fT} \leq L_{fT} & \forall f \in \mathcal{F}\\
		& D_{fT} = HX_{f(T-1)N} & \forall f \in \mathcal{F}\\
		& L_{fT} = HX_{f(T-1)(N+\tau_{f})}& f \in \mathcal{F}\\
		& R_f \geq 1 - (Y_{fT} + Z_{fT}) & f \in \mathcal{F}\\
		& RE_f \geq 1 - (E_{fT} + EZ_{fT}) & f \in \mathcal{F}_3\\
		& HD_{fT} = D_{f(T-1)} & f \in \mathcal{F}\\
		& HE_{fT} = E_{f(T-1)} & f \in \mathcal{F}_3\\
		& HEZ_{fT} = EZ_{f(T-1)} & f \in \mathcal{F}_3\\
		& HX_{fti} = HX_{f(T-1)(i-1)} & f \in \mathcal{F}, i = 2, \dots, \tau_{f} + N\\
		& HY_{fT} = Y_{f(T-1)} & f \in \mathcal{F}\\
		& HZ_{fT} = Z_{f(T-1)} & f \in \mathcal{F}\\
		& \sum_{f \in \mathcal{F}} Y_{fT} - HY_{fT} \leq CA^{\omega_T} &\\
		& \sum_{f \in \mathcal{F}_3} E_{fT} + HE_{fT} \leq CD^{\omega_T} &\\
		& \sum_{f \in \mathcal{F}} (Y_{fT} - HY_{fT}) + \sum_{f \in \mathcal{F}_3} (E_{fT} + HE_{fT}) \leq CT^{\omega_T} &\\
		& X_{fT} \geq HX_{fT} & \forall f \in \mathcal{F}\\
		& Z_{fT} \geq HZ_{fT} & \forall f \in \mathcal{F}\\
		& Y_{fT} \geq HY_{fT} & \forall f \in \mathcal{F}\\
		& E_{fT} \geq HE_{fT} & \forall f \in \mathcal{F}_3\\
		& EZ_{fT} \geq HEZ_{fT} & \forall f \in \mathcal{F}_3\\
		& HX_{fTi} \geq HX_{fT(i+1)} & \forall f \in \mathcal{F}, i = 1, \dots, \tau_{f} + N -1.
		\end{align}
	\end{subequations}
%	In this problem, \(HX,HD,HY,HL,HZ,HE,HEZ\) are all local variables. Constraint Set (\ref{inheritCon}) is used to set up the local copy in the following way:
%	\begin{align*}
%		& HX_f = H_{f,(t-1),1}& \forall f \in \mathcal{F}_1\\
%		& HD_f = H_{f,(t-1),N}& \forall f \in \mathcal{F}_1\\
%		& HY_f = V_{f,(t-1)}& \forall f \in \mathcal{F}\\
%		& HL_f = H_{f,(t-1),N+\tau_{f}}& \forall f \in \mathcal{F}\\
%		& HZ_f = U_{f,(t-1)}& \forall f \in \mathcal{F}_1\\
%		& HE_f = W_{f,(t-1)}& \forall f \in \mathcal{F}_3\\
%		& HEZ_f = U_{f,(t-1)}& \forall f \in \mathcal{F}_3\\
%	\end{align*}
%	To proceed to the next time period \(t+1\), the state variable should be updated using the decisions in the time period \(t\), which is specified in Constraint Set (\ref{passDown}):
%	\begin{align*}
%		& H_{f,t,1} = X_f & \forall f \in \mathcal{F}_1\\
%		& H_{f,t,i} = H_{f,t-1,(i-1)} & \forall f \in \mathcal{F}_1, i = 2, \dots, N+\tau_f\\
%		& V_{f,t} = Y_f & \forall f \in \mathcal{F}\\
%		& W_{f,t} = E_f & \forall f \in \mathcal{F}_3\\
%		& U_{f,t} = Z_f & \forall f \in \mathcal{F}_1\\
%		& U_{f,t} = EZ_f & \forall f \in \mathcal{F}_3
%	\end{align*}
	Since it is impossible to find an analytical representation of the function \(Q_t(x_{t-1},\xi_t), \forall t = 1, \dots, T-1\), we set up a variable \(\theta_t\)  as a surrogate and sequentially generate cuts to approximate this function. The cuts take the form of:
	\begin{equation}
		\theta_t \geq \pi_t^k(x_{t-1} - \hat{x}_{t-1}) + v_t^k, \qquad k = 1, 2, \dots
	\end{equation}
	In the process described above, we are generating a cut for every stage in one iteration. It is possible to generate multiple cuts at the same time since we assume the set of scenarios in one stage is a finite discrete set. We denote the total number of scenarios of stage \(t\) as \(N_t\), \(\forall t = 2,3, \dots T\) and we employ \(N_t\) number of surrogates to represent the value function of each scenario. The multiple cuts we generate in each iteration can be represented as:
	\begin{equation}
		\theta_t^i \geq \pi_t^{k,i}(x_{t-1} - \hat{x}_{t-1}) + v_t^{k,i}, \qquad k = 1, 2, \dots,\ i = 1, \dots, N_t
	\end{equation}
	\textcolor{blue}{Add the relaxation problem to generate the cuts.}\\
	\newline
	The stochastic dual dynamic programming (SDDP) algorithm to solve this multi-stage stochastic integer program is displayed as follows. Every iteration we sample a single path along the scenario tree and generate cuts to better approximate the value function of each stage. Since we cannot enumerate all branches in the scenario tree and get the solution for each of them, it is not possible for us to obtain an exact upper bound \(\overline{z}\). Instead, we calculate a statistical upper bound \(\overline{z}^*\) every \(N\) iteration, with \(N = 20\) in our case. To obtain such a statistical upper bound, we take \(M = 100\) sample paths and solve for the solution along each sample path. Suppose the objective function value of each sample path is \(u_i,\ i = 1, \dots, M\) and \(z_{\alpha/2}\) is the \(1 - \frac{\alpha}{2}\) quantile function value of the standard normal distribution, the \(\alpha\)-level statistical upper bound is:
	\begin{equation}
		\overline{z}^*_\alpha = \mu + z_{\alpha/2} \frac{\frac{1}{M-1}\sum_{i = 1}^{M} \sigma}{\sqrt{M}},
	\end{equation}
	where 
	\[\mu = \frac{1}{M}\sum_{i = 1}^{M} u_i \]
	and
	\[\sigma^2 = (u_i - \frac{1}{M}\sum_{i = 1}^{M} u_i)^2.\]
	\begin{algorithm}[H]
		\caption{SDDP algorithm to solve the multi-stage GDP problem}
		\label{alg:SDDP}
		\begin{algorithmic}[1]
			\State Initialize with iteration number \(k := 1\), \(LB := -\infty\), \(UB = +\infty\), set up the single stage problems \(P_t^k,\ \forall t = 1, \dots, T\) and select parameters \(M, N\) and \(\epsilon\);
			\While{\(\frac{UB - LB}{LB} > \epsilon\)} 
			\State Sample one scenario path \(\Omega = \{\xi_2, \dots, T\}\); \\
			\State /Forward Step/
			\For{\(t := 1 \text{ to } T\)}
			\State Solve problem \(P_t^k\) to collect solution \(x_t^k, \theta_t^k\);
			\EndFor\\
			\State /Update statistical upper bound/\\
			\If{\(k \mod N = 0\)}
			\State Sample \(M\) scenario paths \(\Omega^i = \{\xi^i_2, \dots, \xi^i_T\}, \ \forall i = 1, \dots, M\);
			\For{\(i := 1\) to \(M\)}
			\State Solve problem \(P_t^k\) with scenario path \(i\) to collect solution \(x_t^i\), \(\forall t = 1, \dots, T\);
			\State Calculate \(u^i := \sum_{t = 2}^{T} f(x_t^i, \xi_t^i) + f_1(x_1^i)\);
			\EndFor
			\State Let \(\mu := \frac{1}{M} \sum_{i = 1}^{M} u_i\) and \(\sigma^2 := \frac{1}{M-1} (u_i - \mu)^2\), update the statistical upper bound \(UB := \mu + z_{\alpha/2}\frac{\sigma}{\sqrt{M}}\);
			\EndIf\\
			\State /Backward Step/\\
			\For{\(t := T - 1\) to \(1\)}
			\For{\(j := 1, \dots, N_t\)}
			\State Solve the relaxation problem \(R_t^j\) and collect cut coefficients \((v_t^{k,j}, \pi_t^{k,j})\);
			\EndFor
			\State Add cut(s) to update Problem \(P_t^{k+1}\);
			\EndFor
			\State Update the lower bound using the objective value of \(P_1^{k+1}\);
			\State \(k := k + 1\);
			\EndWhile
		\end{algorithmic}
	\end{algorithm}
	\noindent If we implement a vanilla version of SDDiP to solve this problem, we will solve a problem with 48 stages, with each stage being a large-scale MIP. This may not be computationally feasible in real-world application, as shown in Section~\ref{expResults}. We seek for other techniques to improve the computational performance of our vanilla SDDP algorithm.\\
	\newline
	\textcolor{blue}{Rolling Horizon Scheme}\\
	\textcolor{blue}{Other heuristics to accelerate the process}
\section{Experimental Results}\label{expResults}

\nocite{*}
\bibliographystyle{plain}
\bibliography{gdp_model}

\end{document}
